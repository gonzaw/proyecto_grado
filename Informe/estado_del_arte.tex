%%%%%%%%%%%%%%%%%%%%%%%%%%%%%%%%%%%%%%%%%%%%%%%%%%
% Records Extensibles - Estado del arte
%%%%%%%%%%%%%%%%%%%%%%%%%%%%%%%%%%%%%%%%%%%%%%%%%

\chapter{Estado del arte}
\label{ch:2}
En este capítulo se presetan varias implementaciones de records extensibles en varios lenguajes. El trabajo se enfoca en lenguajes fuertemente tipados. Existen implementaciones de records extensibles en lenguajes con tipado dinámico, pero caen fuera del alcance de este trabajo.

En términos generales, las implementaciones de records extensibles se dividen según si son proporcionadas por el lenguaje como primitivas, o si son proporcionadas por bibliotecas. Se describirán ambas alternativas, incluyendo ejemplos de lenguajes e implementaciones de cada uno.

En particular, este trabajo se enmarca dentro de los proporcionados como bibliotecas de usuario. Este trabajo fue motivado por una biblioteca de Haskell llamada \textit{HList}, la cual se presenta más adelante.

\section{Records extensibles como primitivos}

Algunos lenguajes de programación funcionales permiten el manejo de records extensibles como primitiva del lenguaje. Esto significa que records extensibles son funcionalidades del lenguaje en sí, y tienen sintaxis y funcionamiento especial en el lenguaje.

Uno de ellos es \textit{Elm} \cite{Czaplicki:2013:AFR:2499370.2462161}. Uno de los ejemplos que se muestra en su documentación (\cite{ElmRecords}) es el siguiente:

\begin{code}
type alias Positioned a =
  { a | x : Float, y : Float }

type alias Named a =
  { a | name : String }

type alias Moving a =
  { a | velocity : Float, angle : Float }
\end{code}

Elm permite definir tipos que equivalen a records, pero agregándole campos adicionales, como es el caso de los tipos descritos arriba. Este tipo de record extensible hace uso de \textit{row polymorphism}. Básicamente, al definir un record, éste se hace polimórfico sobre el resto de los campos (o \textit{rows}, como es descrito en la literatura). Es decir, se puede definir un record que tenga como mínimo unos determinados campos, pero el resto de éstos puede variar. Su uso sería el siguiente:

\begin{code}
lady : Named { age : Int }
lady =
  { name = "Lois Lane"
  , age = 31 
  }

dude : Named (Moving (Positioned {}))
dude =
  { x = 0
  , y = 0
  , name = "Clark Kent"
  , velocity = 42
  , angle = degrees 30
  }
\end{code}

En el ejemplo de arriba, el record \texttt{lady} es definido extendiendo \texttt{Named} con otro campo adicional, sin necesidad de definir un tipo nuevo.

Por el poco uso de records extensibles, en la versión 0.16 se decidió eliminar la funcionalidad de agregar y eliminar campos a records de forma dinámica \cite{ElmReducedRecordSyntax}.

Otro lenguaje con esta particularidad es \textit{Purescript} \cite{PurescriptByExample}. A continuación se muestra uno de los ejemplos de su documentación \cite{PurescriptRecords}:

\begin{code}
fullname :: forall t. { firstName :: String,
  lastName :: String | t } -> String
fullname person = person.firstName ++ " " ++ 
  person.lastName
\end{code}

Purescript permite definir records con determinados campos, y luego definir funciones que solo actúan sobre los campos necesarios. Utiliza \textit{row polymorphism} al igual que Elm.

Ambos lenguajes basan su implementación de records extensibles, aunque sea parcialmente, en \cite{Leijen:scopedlabels}. En dicho trabajo Leijen describe un sistema de tipos donde los records extensibles son una nueva extensión del lenguaje, y permite que éstos puedan contener etiquetas repetidas, asignando un \textit{alcance} a cada etiqueta. Cada acceso a un elemento del record mediante una etiqueta simplemente accede al primer elemento de éste con esa etiqueta. La sintaxis y operaciones sobre records que se definen en este paper es la utilizada en Elm y Purescript, como por ejemplo \texttt{\{ l = e | r \}}, que extiende un record \texttt{r} con la etiqueta \texttt{l} y el valor \texttt{e}. Este trabajo implementa \textit{row polymorphism} al designar un \textit{kind} (tipo de tipos) llamado \textit{row} y permitiendo definir tipos de records parametrizados por tales rows.

Existen otras propuestas de sistemas de tipos con soporte para records extensibles. En \cite{Leijen:fclabels}, Leijen describe un sistema de tipos con etiquetas como \textit{first-class citizens} del lenguaje para permitir construir records extensibles de forma expresiva. 

En \cite{Gaster96apolymorphic}, Gaster y Jones también describen un sistema de tipos con soporte para records extensibles. Ambos trabajos se basan en extensiones del sistema de tipos de Haskell y ML, incluyendo inferencia de tipos. El trabajo de Gaster y Jones fue utilizado para la extensión Trex de records extensibles de Hugs98 \cite{TrexHugsExtension} (Hugs98 es un intérprete de Haskell que desafortunadamente ha sido descontinuado).

Otra propuesta es la de \cite{Cardelli:1990:OR:101514.101515}. Esta propuesta desarrolla una teoría de records extensibles, formalizando un sistema de tipos para soportarlos. Otra propuesta es \cite{Jones99lightweightextensible}, de Jones y Peyton Jones, la cual describe una extensión de Haskell98 para soportar records extensibles. Muchas de las propuestas son similares y utilizan los mismos fundamentos de la teoría de records.

Estas propuestas, al igual que cualquier propuesta de extender un lenguaje para que soporte records extensibles como primitivas, tienen algunas desventajas. La desventaja principal es que se deben agregar nuevas reglas de tipado para soportar los records, y esto puede impactar otros aspectos del lenguaje. En \cite{Jones99lightweightextensible}, los autores describen un problema de su extensión de records, donde al tener un nuevo kind (tipo de tipos) sobre records, existen ambigüedades y problemas al integrarlo con el sistema de \textit{typeclasses} de Haskell. Al agregar una nueva regla al lenguaje para soportar records, esta regla puede llegar a ser inconsistente con otras partes del lenguaje, y puede requerir un rediseño de muchas partes de éste para que funcionen correctamente con estas nuevas reglas. Esto también genera que el lenguaje sea más costoso de mantener y entender, lo cual puede traer problemas futuros cuando se intenten implementar nuevas funcionalidades y extensiones del lenguaje. 

Otro problema es que, al menos que se explicite, las nuevas funcionalidades del lenguaje que soportan records extensibles no son \textit{first-class}. En algunas propuestas vistas, el lenguaje requiere de sintaxis especial para el manejo de records (como el kind \texttt{ROW} en Haskell). Esta falta de soporte limita la expresividad de tales records, ya que divide el lenguaje en dos, separando la manipulación del record extensible de la manipulación de otros valores del lenguaje. Esta separación evita que, por ejemplo, se puedan desarrollar funcionalidades parametrizables por ambos, como por ejemplo, desarrollar una función que se pueda parametrizar tanto por un valor común (como uno de tipo \texttt{Int}) como por un record extensible.

Este trabajo se enfoca en el abordaje a records extensibles como bibliotecas de usuario. Una biblioteca no modifica el lenguaje mismo sino que hace uso de todas sus funcionalidades ya definidas, por lo que no presentan los problemas vistos anteriormente. A su vez, las bibliotecas le dan varias opciones al usuario, ya que puede elegir la implementación de records extensibles que más desee y le resulta mejor para su situación.

\section{Records extensibles como bibliotecas de usuario}

Las bibliotecas de usuario son componentes de un lenguaje de programación que están escritas en ese mismo lenguaje, proveyendo una funcionalidad al usuario que utiliza este componente. Las implementaciones de records extensibles que utilizan este mecanismo se basan en utilizar funcionalidades avanzadas del lenguaje en cuestión para poder realizar la definición de tales records.

En \cite{Jeltsch:2010:GRC:1836089.1836108}, Jeltsch describe un sistema de records en Haskell definiendo los records como typeclasses, y los tipos y etiquetas del record como tipos producto. Su trabajo fue liberado como una biblioteca de Haskell llamada \textit{records} \cite{HaskellRecords}.

Otras propuestas se basan en utilizar extensiones de Haskell para poder definir los records. A modo de ejemplo, \textit{rawr} \cite{HaskellRawr} utiliza \textit{type families} y \textit{type-level lists} para definir sus records; \textit{Vinyl} \cite{HaskellVinyl} también utiliza type-level lists, pero define un record como un \textit{GADT} (Generalized Abstract Data Type); \textit{ruin} \cite{HaskellRuin} utiliza \textit{Template Haskell} y metaprogramming para crear instancias de typeclasses específicas. Existen otras propuestas, como \textit{labels} \cite{HaskellLabels}, \textit{named-records} \cite{HaskellNamedRecords}, \textit{bookkeeper} \cite{HaskellBookkeeper}, entre otras. Estas bibliotecas son muy similares en su funcionamiento, difiriendo en qué funcionalidades de GHC (Glasgow Haskell Compiler) y Haskell utilizan, qué funcionalidades de records y records extensibles le proporcionan al usuario, qué sintaxis usan, etc.

Como ejemplo de uso de una biblioteca, se tienen los records de \textit{rawr} que se definen de la siguiente forma:

\begin{code}
type Foo = R ( "a" := Int, "b" := Bool )

foo :: Foo
foo = R ( #a := 42, #b := True ) :: Foo
\end{code}

La definición de un record se realiza llamando a funciones y tipos proporcionados por la biblioteca (como es el caso de las funciones \texttt{R} en este ejemplo). Una diferencia a notar, es que la definición del tipo \texttt{Foo} se utilizaron strings, como \texttt{''a''}, mientras que en la definición de un valor de ese tipo se utilizaron identificadores, como \texttt{\#a}. Esto sucede porque los records en \textit{rawr} hacen uso de \textit{type-level literals}, una funcionalidad del compilador GHC que permite utilizar strings literales en un tipo (llamados \textit{Symbols}). Por lo que \texttt{''a''} en \texttt{R ( ''a'' := Int)} es un string promovido a un tipo. 

Para poder utilizar un valor literal promovido, en GHC se utiliza el tipo \texttt{Proxy t}, el cual solo mantiene una referencia al tipo \texttt{t}. En este caso, \texttt{\#a} es un valor de tipo \texttt{Proxy ''a''}.

Para realizar la extensión de un record, generalmente también se utilizan operadores definidos en la biblioteca, como en el siguiente caso:

\begin{code}
R ( #foo := True) :*: R ( #bar := False)
--> R ( bar := False, foo := True )
\end{code}

El operador \texttt{(:*:)} realiza una unión de dos records. Si uno quiere extender un record \texttt{foo} con una etiqueta \texttt{bar} y un valor \texttt{x}, se puede encapsular los dos últimos en un record y realizar la unión del primero con este último:

\begin{code}
extend foo bar x = foo :*: R ( bar := x)
\end{code}

Para poder obtener un elemento de un record, se puede utilizar su etiqueta directamente como función de acceso:

\begin{code}
rec = R ( #foo := True)
#foo rec
--> True
\end{code}

Entre las muchas propuestas y alternativas, el presente trabajo se basó en la biblioteca HList de Haskell \cite{HListHackage}. HList muestra las problemáticas de definir records extensibles, a la vez que soluciones a ellas. También tiene varias propiedades que son deseadas tener en un diseño de records extensibles (como flexibilidad para agregar nuevas operaciones sobre records). El objetivo del presente trabajo es trasladar las ideas de HList a un contexto de tipos dependientes en Idris y hacer uso de sus buenas propiedades.

Uno de los mecanismos utilizados por HList y la mayoría de las bibliotecas previamente descritas son las \textit{listas heterogéneas}, que se presentan en la siguiente sección. Bibliotecas como HList y Records \cite{HaskellRecords} utilizan funcionalidades más básicas de Haskell para definir listas heterogéneas (como type families y type classes). Algunas de las bibliotecas mencionadas (como \textit{vinyl}) son más modernas y utilizan funcionalidades nuevas de GHC, como \textit{DataKinds}, que permite definir listas heterogéneas a nivel de tipos, y de esa manera es posible definir records extensibles de una forma más sencilla.

Las listas heterogéneas son un pilar base de muchas implementaciones de records extensibles, pero en particular son fundamentales para la implementación de HList, y subsecuentemente para la implementación de records extensibles en Idris. A continuación se describien las listas heterogéneas, luego se muestra cómo se implementan en la biblioteca HList de Haskell, y al final se presenta cómo son implementadas en Idris.

\section{Listas heterogéneas}

El concepto de lista heterogénea (o \textit{HList}) surge en oposición al tipo de listas usualmente utilizado en lenguajes tipados: listas homogéneas. Las listas homogéneas son listas que pueden contener elementos de un solo tipo.

Estos tipos de listas existen en todos o casi todos los lenguajes de programación que aceptan tipos parametrizables o genéricos, sea Java, C\#, Haskell y otros. Ejemplos comunes son \texttt{List<int>} en Java, o \texttt{[Int]} en Haskell.

Las listas heterogéneas, sin embargo, permiten almacenar elementos de distintos tipos. Estos tipos pueden no tener una relación entre ellos.

En lenguajes dinámicamente tipados (como lenguajes basados en LISP) este tipo de listas es fácil de construir, ya que no hay una imposición por parte del intérprete (o compilador) sobre qué tipo de elementos se pueden insertar o pueden existir en esta lista. El siguiente es un ejemplo de lista heterogénea en un lenguaje LISP como Clojure o Scheme, donde a la lista se le puede agregar un entero, un float y un texto:

\begin{code}
'(1 0.2 "Text")
\end{code}

En lenguajes fuertemente tipados este tipo de lista es más difícil de construir. Tales listas heterogéneas pueden tener elementos con tipos arbitrarios, por lo tanto los sistemas de tipos de estos lenguajes necesitan una forma de manejar tal conjunto arbitrario de tipos. No todos los lenguajes fuertemente tipados lo permiten.

En sistemas de tipos más avanzados es posible incluir la información de tales tipos arbitrarios en el mismo tipo de la lista heterogénea.

A continuación se describe la forma de crear listas heterogéneas en Haskell, y luego la forma de crearlas en Idris, utilizando el poder de tipos dependientes de este lenguaje para llevarlo a cabo.

\subsection{HList en Haskell}

Para definir listas heterogéneas en Haskell nos basaremos en la propuesta presentada en \cite{Kiselyov:2004:STH:1017472.1017488}, por Kiselyov, L\"{a}mmel y Schupke. Esta propuesta está implementada en el paquete \textit{hlist} de Hackage \cite{HListHackage}. Su implementación es muy similar a la de la biblioteca \textit{records} descrita anteriormente.

Esta biblioteca define HList de la siguiente forma:

\begin{code}
data family HList (l::[*])

data instance HList '[] = HNil
data instance HList (x ': xs) = x `HCons` HList xs
\end{code}

Se utilizan \textit{data families} para poder crear una familia de tipos \texttt{HList}, que toma una lista de tipos como parámetro, definiendo un caso cuando una lista de tipos es vacía y otro caso cuando no lo es.

\textit{Data families} es una extensión de Haskell, junto a \textit{type synonym families} (juntas denominadas \textit{type families}), que permiten realizar polimorfismo ad-hoc sobre tipos. Cada type family se parametriza sobre un tipo, y permite generar instancias para tipos específicos.

Las instancias de una \textit{data family} indican que, para cada instancia en particular, ésta se construye con un constructor específico y ningún otro. Como ejemplo:

\begin{code}
data family MyData a
data instance MyData Int = 
  MyDataC1 Int | MyDataC2 String
data instance MyData String = MyDataC3 [Int]
\end{code}

Si en algún momento se tiene \texttt{MyData Int}, entonces es posible realizar pattern matching sobre él con los constructores \texttt{MyDataC1} y \texttt{MyDataC2}, pero \textit{no} con \texttt{MyDataC3}. Si se tiene \texttt{MyData String} ocurre lo contrario, se puede realizar pattern matching con el constructor \texttt{MyDataC3} pero no con los demás. Si se tiene \texttt{MyData a} parametrizado por un \texttt{a} arbitrario, no se puede realizar pattern matching hasta conocer más sobre el tipo \texttt{a}.

A diferencia de \textit{data families}, las instancias de un \textit{type synonym family} indican que, para cada instancia en particular, ésta es un sinónimo de tipo del tipo indicado a la derecha. Como ejemplo:

\begin{code}
type family MyFam a
type instance MyFam Int = Maybe Int
type instance MyFam String = [Int]
\end{code}

En este caso, cada vez que se tiene \texttt{MyFam Int}, éste funciona como un alias para \texttt{Maybe Int}. Cada vez que se tiene \texttt{MyFam String}, éste funciona como un alias de \texttt{[Int]}. Si se tiene \texttt{MyFam a} parametrizado por un \texttt{a} arbitrario, no se puede conocer su sinónimo hasta tener más información sobre el tipo \texttt{a}.

Para el caso de \texttt{HList t}, ésta se define como un data family de dos constructores, \texttt{HNil} y \texttt{HCons}, donde \texttt{HNil} solo puede aplicarse cuando la lista \texttt{t} es vacía, y \texttt{HCons} cuando ésta no es vacía. Esto permite, por ejemplo, construir el siguiente valor:

\begin{code}
10 `HCons` ("Text" `HCons` HNil) :: HList '[Int, String]
\end{code}

La estructura recursiva de \texttt{HList} garantiza que es posible construir elementos de este tipo a la misma vez que se van construyendo elementos de la lista de tipos.

Para poder definir funciones sobre este tipo de listas, es necesario utilizar \textit{type synonym families}, como muestra el siguiente ejemplo:

\begin{code}
type family HLength (x :: [k]) :: HNat
type instance HLength '[] = HZero
type instance HLength (x ': xs) = HSucc (HLength xs)

hLength   :: HLengthEq l n => HList l -> Proxy n
hLength _ =  Proxy
\end{code}

Type families como la anterior permiten que se tenga un tipo \texttt{HLength ls} en la definición de una función y que el typechecker decida cuál de las instancias anteriores debe llamar, culminando en un valor del kind \texttt{HNat}.

Luego, la función \texttt{hLength} toma una lista heterogénea, calcula su largo con \texttt{HLengthEq} (cuya implementación no se verá, pero se basa en \texttt{HLength}) y lo retorna en un tipo \texttt{Proxy n} (un \textit{phantom type} que solo mantiene una referencia a \texttt{n}, el largo de la lista, en su tipo).

Un ejemplo de uso de tal función sería el siguiente:

\begin{code}
hLength (10 `HCons` ("Texto" `HCons` HNil)) = 
  Proxy :: Proxy (HSucc (HSucc HZero))
\end{code}

\subsection{HList en Idris}

Uno de los objetivos de la investigación de listas heterogéneas en un lenguaje como Idris es poder utilizar el poder de su sistema de tipos. La intención es poder manipular listas heterogéneas con la misma facilidad que uno lo hace con listas homogéneas. Esto no ocurre en el caso de HList en Haskell, ya que el tipo de HList es definido con data families, y las funciones sobre HList requieren ser definidas con type families y typeclasses. Esta definición se contrasta con las listas homogéneas, cuyo tipo se define como un tipo común, y funciones sobre tales se definen como funciones normales también.

A diferencia de Haskell, Idris maneja tipos dependientes. Esto significa que cualquier tipo puede estar parametrizado por un valor, y en tal caso el tipo es un ciudadano de primera clase que puede ser utilizado como cualquier otro elemento del lenguaje. Por lo tanto, para Idris no hay diferencia en el trato de un tipo simple como \texttt{String} o un tipo complejo con tipos dependientes como \texttt{Vect 2 Nat} (este tipo se verá a continuación).

Para conocer mejor el uso de tipos dependientes en Idris, veamos la definición del tipo \textit{vector}, que representa listas cuyo largo es anotado en su tipo:

\begin{code}
data Vect : Nat -> Type -> Type where
  [] : Vect 0 A
  (::) : (x : A) -> (xd : Vect n A) -> Vect (n + 1) A
\end{code}

Un valor del tipo \texttt{Vect 1 String} es una lista que contiene 1 string, mientras que un valor del tipo \texttt{Vect 10 String} es una lista que contiene 10 strings. En la definición \texttt{Vect : Nat -> Type -> Type} el tipo mismo queda parametrizado por un natural, además de un tipo cualquiera.

Tener valores en el tipo permite poder restringir el uso de funciones a determinados tipos que tengan valores específicos. Como ejemplo se tiene la siguiente función:

\begin{code}
head : Vect (n + 1) a -> a
head (x :: xs) = x
\end{code}

Esta función obtiene el primer valor de un vector. Es una función total, ya que restringe su uso solamente a vectores que tengan un largo mayor a 0, por lo que el vector al que se le aplique esta función siempre va a tener por lo menos un elemento para obtener. Si se intenta llamar a esta función con un vector sin elementos, como \texttt{head []}, la llamada no va a compilar porque el typechecker no va a poder unificar el tipo \texttt{Vect 0 a} con \texttt{Vect (n + 1) a} (no puede encontrar un valor natural \texttt{n} que cumpla \texttt{n + 1 = 0}, por lo que falla el typechecking).

En cuanto a listas heterogéneas, la definición de HList utilizada en Idris es la siguiente:

\begin{code}
data HList : List Type -> Type where
  Nil : HList []
  (::) : t -> HList ts -> HList (t :: ts)
\end{code}

Esta definición permite construir listas heterogéneas con relativa facilidad, como por ejemplo:

\begin{code}
23 :: "Hello World" :: [1,2,3] :: Nil :
  HList [Nat, String, [Nat]]
\end{code}

El tipo \texttt{HList} se define como un tipo indexado por una lista de tipos (e.j \texttt{[Nat, String]}). Éste se construye definiendo una lista vacía que no tiene tipos, o definiendo un operador de \textit{cons} que tome un valor, una lista previa, y agregue ese valor a la lista. En el caso de \textit{cons}, no solo agrega el valor a la lista, sino que agrega el tipo de tal valor a la lista de tipos que mantiene \texttt{HList} en su tipo.

Cada valor agregado a la lista tiene un tipo asociado que es almacenado en la lista del tipo. Por ejemplo, al agregar \texttt{23 : Nat} se guarda \texttt{23} en la lista y \texttt{Nat} en el tipo, de forma que uno siempre puede recuperar ya sea el tipo o el valor si se quieren utilizar luego.

A su vez, a diferencia de Haskell, el hecho de que los tipos dependientes son ciudadanos de primera clase permite definir funciones con pattern matching sobre HList:

\begin{code}
hLength : HList ls -> Nat
hLength Nil = 0
hLength (x :: xs) = 1 + (hLength xs)
\end{code}

Aquí surge una diferencia muy importante con la implementación de HList de Haskell, ya que en Idris \texttt{hLength} puede ser utilizada como cualquier otra función, y en especial opera sobre \textit{valores}, mientras que en Haskell \texttt{HLength} solo puede ser utilizada como type family, y por lo tanto solo opera a nivel de \textit{tipos}. En cambio, el uso de HList en Idris no es distinto del uso de listas comunes (\texttt{List}), y por lo tanto pueden ser tratadas de la misma forma tanto por quienes crean la biblioteca que por quienes la utilizan.

A continuación se presenta cómo la biblioteca de Haskell HList define records extensibles utilizando estas listas heterogéneas.

\section{Records extensibles en Haskell}

La propuesta de HList \cite{Kiselyov:2004:STH:1017472.1017488} utiliza listas heterogéneas para definir un record:

\begin{code}
newtype Record (r :: [*]) = Record (HList r)

mkRecord :: HRLabelSet r => HList r -> Record r
mkRecord = Record
\end{code}

Un record se representa simplemente como una lista heterogénea de un tipo determinado (que se verá más adelante). Un record se puede construir solamente utilizando \texttt{mkRecord}. Esta función toma una lista heterogénea, pero fuerza a que tenga una instancia de \texttt{HRLabelSet}.

Una lista heterogénea con una instancia de \texttt{HRLabelSet} implica que la lista tiene valores con etiquetas, y ninguna etiqueta se repite. Se define de la siguiente forma:

\begin{code}
class (HLabelSet (LabelsOf ps), HAllTaggedLv ps) =>
  HRLabelSet (ps :: [*])
\end{code}

Para poder implementar una instancia de esta typeclass, la lista necesita cumplir el predicado \texttt{HAllTaggedLV} y \texttt{HLabelSet}. Para esto la lista debe contener valores de este tipo:

\begin{code}
data Tagged s b = Tagged b
\end{code}

Este tipo permite tener un \textit{phantom type} en el tipo \texttt{s}. Esto significa que un valor de tipo \texttt{Tagged s b} va a contener solamente un valor de tipo \texttt{b}, pero en tiempo de compilación se va a tener el tipo \texttt{s} para manipular.

El predicado \texttt{HAllTaggedLV} simplemente verifica que la lista solo contenga elementos del tipo \texttt{Tagged}. Ambos \texttt{Tagged} y \texttt{HAllTaggedLV} pertenecen a la biblioteca \textit{tagged} \cite{HaskellTagged}.
 
Estas etiquetas no deben repetirse, y para ello la lista debe poder tener una instancia de \texttt{HLabelSet (LabelsOf ps)}. \texttt{LabelsOf} es una type family que toma una lista de elementos de tipo \texttt{Tagged} y obtiene sus etiquetas:

\begin{code}
type family LabelsOf (ls :: [*]) :: [*]
type instance LabelsOf '[] = '[]
type instance LabelsOf (Label l ': r) = 
  Label l ': LabelsOf r
type instance LabelsOf (Tagged l v ': r) = 
  Label l ': LabelsOf r
\end{code}

\texttt{LabelsOf ls} toma todos los \textit{phantom types} de \texttt{Tagged} y los retorna como etiquetas \texttt{Label l}. Como en algunas partes de la implementación se permite que la lista heterogénea tenga solo etiquetas (sin valores), la lista original también puede tener un valor del tipo \texttt{Label l}. El tipo \texttt{Label l} representa una etiqueta a nivel de tipos para poder identificar los campos del record. Su definición es idéntica a la de \texttt{Proxy} descrita en secciones anteriores, y es la siguiente:

\begin{code}
data Label l = Label
\end{code}

\texttt{HLabelSet} es una typeclass que representa el predicado de que las etiquetas de la lista no estén repetidas. Para ello se aplica la función \texttt{LabelsOf} antes, para poder obtener solo las etiquetas de la lista (y no sus valores). Este predicado se define de forma recursiva definiendo instancias para cada caso base y paso inductivo. Hace uso de predicados de igualdad de tipos para poder realizarlo. Su definición es la siguiente:

\begin{code}
class Fail
data DuplicatedLabel l

class HLabelSet ls
instance HLabelSet '[]
instance HLabelSet '[x]
instance (HEqK l1 l2 leq, 
  HLabelSet' l1 l2 leq r,
  ) => HLabelSet (l1 ': l2 ': r)

class HLabelSet' l1 l2 (leq :: Bool) r
instance (HLabelSet (l2 ': r),
  HLabelSet (l1 ': r),
  ) => HLabelSet' l1 l2 False r
instance (Fail (DuplicatedLabel l1)) =>
  HLabelSet' l1 l2 True r
\end{code}

Como casos base, se define que \texttt{'[]} y \texttt{'[x]} son conjuntos. Para el caso recursivo, se hace uso de una typeclass auxiliar \texttt{HLabelSet' l1 l2 (leq :: Bool) r}. Se fuerza a que \texttt{l1} y \texttt{l2} sean distintos utilizando la igualdad a nivel de tipos \texttt{HEqK l1 l2 leq} y la segunda instancia de \texttt{HLabelSet'}. Si \texttt{l1} y \texttt{l2} fueran iguales, entonces \texttt{HEqK l1 l2 True} sería la instancia obtenida, la cual unificaría con \texttt{HLabelSet' l1 l2 True r}. En este caso, esto fallaría en tiempo de compilación ya que utilizaría la segunda instancia de \texttt{HLabelSet'}, la cual necesita de una instancia de \texttt{Fail (DuplicatedLabel l1)}, que no existe porque la typeclass \texttt{Fail} se define sin instancias (es una typeclass para indicar casos de error). Esto fuerza a que \texttt{l1} y \texttt{l2} sean distintos. En tal caso, si se cumple que \texttt{(l1 ': r)} y \texttt{(l2 ': r)} son conjuntos sin elementos repetidos, se puede concluir que \texttt{(l1 ': l2 ': r)} no tiene elementos repetidos, representado por la instancia \texttt{HLabelSet (l1 ': l2 ': r)}.

La biblioteca también proporciona otra forma de generar records utilizando etiquetas y operadores especiales. A continuación mostraremos un ejemplo de su uso, adaptando el ejemplo visto en el paper de HList :

Teniendo las etiquetas \texttt{clave}, \texttt{nombre}, y \texttt{edad} previamente definidas, la biblioteca permite definir registros de la siguiente manera:

\begin{code}
persona = clave .=. 3
  .*. nombre .=. "Juan"
  .*. edad .=. 27
  .*. emptyRecord
\end{code}

Los campos del record se definen utilizando el siguiente operador:

\begin{code}
(.=.) :: Label l -> v -> Tagged l v
\end{code}

A su vez, el operador \texttt{(.*.)} permite extender un record con un nuevo campo. Su tipo es el siguiente:

\begin{code}
(.*.) :: HExtend e l => e -> l -> HExtendR e l
\end{code}

\texttt{HExtendR e l} es una type family que representa el resultado final de extender el record \texttt{e} con el campo \texttt{l}. A su vez, la typeclass \texttt{HExtend} funciona como un predicado que indica que es posible extender \texttt{e} con \texttt{l}. Su implementación no se mostrará, pero cabe notar que la misma termina realizando un llamado a la función \texttt{mkRecord} vista anteriormente.

En HList, dos etiquetas \texttt{Label (Lbl x1 n1 desc)} y \texttt{Label (Lbl x2 n2 desc)} son iguales si sus namespaces y posiciones son idénticas. Es decir, si \texttt{x1 = x2} y \texttt{n1 = n2}. Por este motivo el chequeo de etiquetas repetidas y su búsqueda se realiza tomando valores de \texttt{Label} y verificando su posición en la lista de etiquetas creadas con \texttt{firstLabel} y \texttt{nextLabel}. Los textos en sí (como el string \texttt{'clave'}) no son utilizados.

Todas las funciones sobre records extensibles se definen utilizando typeclasses para realizar la computación en los tipos. Un ejemplo de ello es la función que obtiene un elemento de un record dada su etiqueta:

\begin{code}
class HasField (l :: k) r v | l r -> v where
  hLookupByLabel :: Label l -> r -> v

(.!.) :: (HasField l r v) => r -> Label l -> v
r .! l = hLookupByLabel l r
\end{code}

Esta typeclass hace uso de \textit{dependencias funcionales}. Las dependencias funcionales son una extensión de Haskell que permiten definir typeclasses donde algunos de sus argumentos determinan de forma única el resto. En el caso anterior, la typeclass \texttt{HasField} tiene 3 dependencias, \texttt{l}, \texttt{r} y \texttt{v}. Sin embargo, tiene una dependencia \texttt{l r -> v} que indica que solo puede existir una única instancia de \texttt{HasField} para cada par de \texttt{l} y \texttt{r}. Su concepto es análogo al de dependencias funcionales en bases de datos relacionales.

La función \texttt{hLookupByLabel} toma una etiqueta \texttt{Label l} y verifica si ésta existe en el record \texttt{r}. Cuando encuentra la etiqueta, retorna el valor que se encuentra en la estructura \texttt{Tagged}.

Para este trabajo se tomó como motivación esta implementación de records extensibles, realizando una traducción a Idris de cada función, tipo y algoritmo. Una diferencia de este trabajo con HList es que en Idris se realizó la comparación de etiquetas comparando por el texto de las mismas, mientras que en HList éstas se comparan utilizando su posición y namespace.

En el siguiente capítulo se muestra la implementación realizada en este trabajo.
