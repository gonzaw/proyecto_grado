%%%%%%%%%%%%%%%%%%%%%%%%%%%%%%%%%%%%%%%%%%%%
% Records Extensibles en Idris
%%%%%%%%%%%%%%%%%%%%%%%%%%%%%%%%%%%%%%%%%%%%

\chapter{Records extensibles en Idris}
\label{ch:3}

En este capítulo se describe cómo se implementaron los records extensibles en Idris. Se comienza mostrando ejemplos de creación y uso de records. Luego se explica el diseño de los records y cuáles funcionalidades de Idris lo hicieron posible. En otra sección se muestra la implementación de algunas de las funciones sobre records extensibles. No se muestran sus implementaciones completas, sino que se describen los rasgos más importantes de la implementación, dejando los detalles y funciones auxiliares para ver en el apéndice. El capítulo termina comparando esta solución de records extensibles con otras vistas en el capítulo \ref{ch:2}.

\section{Introducción a records extensibles en Idris}

En este trabajo se decidió seguir el diseño de HList de Haskell para extender records. Como ejemplo, podemos tomar el siguiente caso de HList descrito en el capítulo anterior:

\begin{code}
persona = clave .=. 3
  .*. nombre .=. "Juan"
  .*. edad .=. 27
  .*. emptyRecord
\end{code}

Este caso puede expresarse en Idris de la siguiente forma:

\begin{code}
persona : Record [("Clave", Nat), ("Nombre", String),
  ("Edad", Nat)]
persona = consRecAuto "Clave" 3 $
  consRecAuto "Nombre" "Juan" $
  consRecAuto "Edad" 27 $
  emptyRec
\end{code}

Las funciones y valores utilizados son comparables con las de HList. La diferencia entre ambos es que Haskell infiere el tipo de \texttt{persona} pero Idris no.

\begin{code}
persona : Record [("Clave", Nat), ("Nombre", String),
  ("Edad", Nat)]
\end{code}

El tipo base de este trabajo es \texttt{Record}. Como se ve, \texttt{Record} contiene en su tipo la lista \texttt{[(''Clave'', Nat), (''Nombre'', String), (''Edad'', Nat)]}. Esta lista representa los campos (o \textit{rows}) del record. Está formada por tuplas, donde el primer valor de la tupla es la etiqueta del campo, y el segundo valor es el tipo del campo. Aquí se aprecia el uso de tipos dependientes, ya que el tipo \texttt{Record} depende de un valor (una lista en particular).

En Idris el tipo de \texttt{Record} es \texttt{Record : List (String, Type) -> Type}. Representa una función de tipo, que se puede aplicar a un valor y retornar un tipo nuevo, como \texttt{Record [(''Clave'', Nat)] : Type}.

Dentro del tipo del record se encuentra toda la información necesaria, ya que se encuentran etiquetas y el tipo de cada campo. No es necesario definir valores externos al record como las etiquetas \texttt{clave} de HList (de tipo \texttt{Label}), sino que se pueden usar simplemente tipos como \texttt{String} para definir las etiquetas.

Otra pieza fundamental en este trabajo es \texttt{consRecAuto}, que es la función que permite tomar un record y extenderlo con otro campo. Consideremos el siguiente ejemplo:

\begin{code}
persona2 : Record [("Nombre", String), ("Edad", Nat)]
persona2 = consRecAuto "Nombre" "Juan" $
  consRecAuto "Edad" 27 $
  emptyRec
\end{code}

Una vez que tenemos este record podemos ahora extenderlo con un campo más:

\begin{code}
persona1 : Record [("Clave", Nat), ("Nombre", String),
  ("Edad", Nat)]
persona1 = consRecAuto "Clave" 3 persona2
\end{code}

El nuevo record, además de tener el campo adicional, tiene anotado en su tipo la tupla \texttt{(''Clave'', Nat)} correspondiente a ese nuevo campo. El tipo de \texttt{consRecAuto} es el siguiente:

\begin{code}
consRecAuto : {A : Type} -> (l : String) -> (a : A) ->
  Record ts -> Record ((l, A) :: ts)
\end{code}

En Idris los paréntesis \texttt{\{\}} indican argumentos implícitos mientras que \texttt{()} indican argumentos explícitos. El compilador intenta inferir los argumentos implícitos conociendo cuáles otros argumentos fueron pasados a la función, aunque a veces no puede hacerlo y es necesario pasarlos explícitamente. En este caso, \texttt{consRecAuto ''Clave'' 3 persona2} es equivalente a \texttt{consRecAuto \{A = Nat\} ''Clave'' 3 persona2}.

Inductivamente, se puede seguir aplicando el razonamiento anterior para deducir el tipo de \texttt{emptyRec}:

\begin{code}
emptyRec : Record []
\end{code}

Además de la operación para la extensión de records, en este trabajo se definieron operaciones para la creación, manipulación y lookup de records. A modo de ejemplo, se tienen estos dos records, que representan distintos atributos de una persona:

\begin{code}
personaConNombre : Record [("Clave", Nat), ("Nombre", String)]
personaConNombre = consRecAuto "Clave" 1 $
  consRecAuto "Nombre" "Juan" $
  emptyRec
-- { "Clave": 1, "Nombre": "Juan" }

personaConEdad : Record [("Clave", Nat), ("Edad", Nat)]
personaConEdad = consRecAuto "Clave" 2 $
  consRecAuto "Edad" 34 $
  emptyRec
-- { "Clave": 2, "Edad": 34 }
\end{code}

Estos records se pueden unir de la siguiente forma:

\begin{code}
persona : Record [("Clave", Nat), ("Nombre", String),
  ("Edad", Nat)]
persona = hLeftUnion personaConNombre personaConEdad
-- { "Clave": 1, "Nombre": "Juan", "Edad": 34 }
\end{code}

Esta unión unifica los campos de ambos, quedándose con el valor de la izquierda para los casos de campos repetidos (como en el caso de \textit{''Clave''}).

Se puede obtener el valor de cualquiera de sus campos:

\begin{code}
nombre : String
nombre = hLookupByLabelAuto "Nombre" persona
-- "Juan"
\end{code}

También se puede actualizar un campo individualmente:

\begin{code}
personaActualizada : Record [("Clave", Nat), ("Nombre", String),
  ("Edad", Nat)]
personaActualizada = hUpdateAtLabelAuto "Nombre" "Pedro" persona
-- { "Clave": 1, "Nombre": "Pedro", "Edad": 34 }
\end{code}

Si se tiene un record con valores completamente distintos, también se pueden añadir al final del record original:

\begin{code}
direccion : Record [("Direccion", String)]
direccion = consRecAuto "Direccion" "18 de Julio" $
  emptyRec
-- { "Direccion" : "18 de Julio" }

personaYDireccion : Record [("Clave", Nat), ("Nombre", String),
  ("Edad", Nat), ("Direccion", String)]
personaYDireccion = hAppendAuto persona direccion
-- { "Clave": 1, "Nombre": "Juan", "Edad": 34,
  "Direccion": "18 de Julio" }
\end{code}

Si se tiene un record, se puede obtener un subrecord proyectando por alguno de sus campos:

\begin{code}
proyeccion : Record [("Clave", Nat), ("Direccion", String)]
proyeccion = hProjectByLabelsAuto ["Clave", "Direccion"]
  personaYDireccion
-- { "Clave": 1, "Direccion": "18 de Julio" }
\end{code}

Todas estas funciones fueron traducidas de la biblioteca HList de Haskell. La implementación y descripción de ellas se ve más adelante.

\section{Predicados y propiedades en Idris}

Antes de ver la implementación de \texttt{consRecAuto} y \texttt{emptyRec}, es necesario entender la otra propiedad de estos records. Al igual que en HList, se desea que en tiempo de compilación se sepa que las etiquetas del record no son repetidas. En particular, para \texttt{consRecAuto}, se necesita saber que la etiqueta nueva a agregar no existe actualmente en el record. Por lo tanto, se necesita más información del record y del campo a agregar para poder extender un record.

La función que extiende un record debería tener el siguiente tipo:

\begin{code}
consRec : {A : Type} -> (l : String) -> (a : A) ->
  (prf : Not (Elem l ts)) -> Record ts -> 
  Record ((l, A) :: ts)
\end{code}

Notar que se agregó el término \texttt{prf : Not (Elem l ts)}. En Idris esto representa no solo un valor \texttt{prf} de un tipo en particular, sino que representa una prueba del predicado \texttt{Not (Elem l ts)} que representa la proposición \textit{''La etiqueta 'l' no pertenece a la lista de campos 'ts' ''}.

Esta correspondencia entre tipos y proposiciones se conoce como isomorfismo de \textit{Curry-Howard} \cite{Howard80}.

El predicado \texttt{Elem} puede definirse inductivamente como cualquier otro tipo de datos, declarando sus constructores:

\begin{code}
data Elem : a -> List a -> Type where
  Here : Elem x (x :: xs)
  There : Elem x xs -> Elem x (y :: xs)
\end{code}

El término \texttt{Elem x xs} representa el predicado \textit{''El elemento 'x' se encuentra en la lista 'xs' ''}. La definición es inductiva, se tiene el caso base en \texttt{Here} y el caso inductivo en \texttt{There}. El caso base ocurre cuando el elemento a comparar es idéntico al primer elemento de la lista. El caso inductivo ocurre cuando se busca el elemento en el resto de la lista.

La construcción de un término del tipo \texttt{Elem x xs} constituye entonces una prueba de que \texttt{x} pertenece a la lista \texttt{xs}. Por ejemplo:

\begin{code}
Here : Elem 3 [3]
There Here : Elem 3 [4, 3]
There (There Here) : Elem 3 [1, 4, 3]
\end{code}

En el tipo de \texttt{consRec}, se tenía el tipo \texttt{Not (Elem l ts)}. \texttt{Not} es una función entre tipos que toma un tipo y retorna otro, representando la negación de un predicado. Tener un valor \texttt{prf : Not (Elem l ts)} significa que es imposible obtener una prueba de \texttt{Elem l ts}. La definición de \texttt{Not} es la siguiente:

\begin{code}
Not : Type -> Type
Not t = t -> Void
\end{code}

\texttt{Not} es una simple función, que toma un tipo y retorna el tipo  \texttt{Void}, que es un tipo sin ninguna prueba, o sea, un tipo al que no se le puede construir ningún valor ya que no tiene constructores. En lógica constructiva, representa el valor \textit{False}, tal que si se puede obtener una prueba de ese valor, entonces se puede obtener una prueba de cualquier otro valor arbitrario. En Idris esa regla está representada por una función proporcionada por el lenguaje llamada \texttt{absurd}:

\begin{code}
absurd : {a : Type} -> Void -> a
\end{code}

Si en algún momento se tiene un valor \texttt{v : Void}, entonces siempre se puede obtener un valor de cualquier tipo con \texttt{absurd v : a}, sea el tipo que sea. Esta función es generalmente usada cuando se está en un caso de pattern matching 'imposible', y se quiere probar que es imposible que la ejecución del programa llegue a ese caso, por lo que se prueba \texttt{Void} y luego se aplica \texttt{absurd} para convencer al compilador.

En Idris, la forma más directa de crear pruebas de \texttt{Void} es mediante la imposibilidad de aplicar constructores. Si al hacer inducción o pattern matching sobre un valor es imposible encontrar un constructor que retorne un tipo compatible con él, entonces se puede utilizar el término \texttt{impossible} y crear una prueba de \texttt{Void} (básicamente, se encuentra un término que debería ser imposible de construir, por lo que se probó el absurdo).

Un ejemplo simple es el siguiente:

\begin{code}
noEmptyElem : Elem x [] -> Void
noEmptyElem Here impossible
\end{code}

En este ejemplo, si se tiene un valor de tipo \texttt{Elem x []}, y si el mismo se hubiera construido con \texttt{Here}, entonces debería tener un tipo que unifique con \texttt{Elem x (x :: xs)}, algo que es imposible ya que la lista proporcionada es vacía. De la misma forma, si se hubiera construido con \texttt{There}, entonces debería tener un tipo que unifique con \texttt{Elem x (y :: xs)}, lo cual tiene el mismo problema. Por lo tanto es imposible construir un valor del tipo \texttt{Elem x []}, por lo que si se tiene tal valor, al aplicarle la función \texttt{noEmptyElem} se puede probar Void.

En Idris, para generar una prueba de \texttt{Void}, basta con incluir un solo constructor (como \texttt{Here} en este caso).

Con la definición de \texttt{Not} vista anteriormente, se puede reescribir esta función de la siguiente forma:

\begin{code}
noEmptyElem : Not (Elem x [])
noEmptyElem Here impossible
\end{code}

Esta forma es la forma más directa de construir pruebas de \texttt{Not}.

Volviendo al caso de records extensibles, en las siguientes secciones se muestra la definición completa de la extensión de records. Sin embargo, primero mostraremos algunas funciones, tipos, nomenclatura y conceptos necesarios para poder hacerlo.

En primer lugar, describiremos qué son los tipos decidibles y para qué son utilizados en este trabajo.

\subsection{Tipos decidibles}

Para poder definir los records extensibles en Idris, es necesario trabajar con tipos o predicados decidibles. En Idris la decidibilidad de un tipo se representa con el siguiente tipo:

\begin{code}
data Dec : Type -> Type where
  Yes : (prf : prop) -> Dec prop
  No : (contra : Not prop) -> Dec prop
\end{code}

Un tipo es decidible si se puede construir un valor de sí mismo, o se puede construir un valor de su negación. Si se tiene \texttt{Dec P}, entonces significa que o bien existe un valor \texttt{s : P} o existe un valor \texttt{n : Not P}.

Tener tipos decidibles es importante cuando se tienen tipos que funcionan como predicados y se necesita saber si ese predicado se cumple o no. Teniendo una prueba de \texttt{Dec P} se puede entonces realizar un análisis de casos, uno cuando \texttt{P} es verdadero y otro cuando no.

Otra funcionalidad importante es la de poder realizar igualdad de valores:

\begin{code}
interface DecEq t where
  total decEq : (x1 : t) -> (x2 : t) -> Dec (x1 = x2)
\end{code}

En idris \texttt{interface} permite definir una abstracción igual a las typeclasses de Haskell, donde se parametriza por un tipo y luego se pueden construir instancias de esa interfaz/typeclass para tipos particulares.

\texttt{DecEq t} indica que, siempre que se tienen dos elementos \texttt{x2, x2 : t}, es posible tener una prueba de que son iguales o una prueba de que son distintos. La función \texttt{decEq} es importante cuando se quiere realizar un análisis de casos sobre la igualdad de dos elementos, un caso donde son iguales y otro caso donde se tiene una prueba de que no lo son.

Los tipos decidibles y las funciones que permiten obtener valores del estilo \texttt{Dec P} son muy importantes al momento de probar teoremas y manipular predicados. En este trabajo los tipos decidibles son principalmente utilizados para generalizar las etiquetas a tipos que no sean String (incluso si en este trabajo principalmente se trabaja con strings). Su uso es bastante simple, en vez de tener \texttt{List (String, Type)} se tiene \texttt{DecEq lty => List (lty, Type)}. Con esta definición es posible utilizar etiquetas de otros tipos, como enumerados, naturales, etc.

\subsection{Funciones y tipos útiles}

Otras funciones y tipos útiles para este trabajo son los siguientes:

\begin{code}
LabelList : Type -> Type
LabelList lty = List (lty, Type)
\end{code}

\texttt{LabelList} es una abstracción que representa una lista de campos con etiquetas y tipos. Por ejemplo, \texttt{[(''Clave'', Nat)] : LabelList String}.

\begin{code}
labelsOf : LabelList lty -> List lty
labelsOf = map fst
\end{code}

La función \texttt{labelsOf} toma una lista de campos y retorna la lista de sus etiquetas. Por ejemplo, \texttt{labelsOf [(''Clave'', Nat), (''Edad'', Nat)] = [''Clave'', ''Edad'']}.

\begin{code}
ElemLabel : lty -> LabelList lty -> Type
ElemLabel l ts = Elem l (labelsOf ts)
\end{code}

\texttt{ElemLabel} es una abreviación para el tipo de la prueba de que una etiqueta pertenece a una lista de campos.

\begin{code}
isElemLabel : DecEq lty => (l : lty) ->
  (ts : LabelList lty) ->
  Dec (ElemLabel l ts)
isElemLabel l ts = isElem l (labelsOf ts)
\end{code}

\texttt{isElemLabel} es una función de decisión, donde para cualquier etiqueta y cualquier lista, se puede probar si tal etiqueta pertenece a esa lista o no. Se basa en una función de decisión ya existente en las bibliotecas base de Idris llamada \texttt{isElem}:

\begin{code}
isElem : DecEq a => (x : a) -> (xs : List a) ->
  Dec (Elem x xs)
\end{code}

Con estos conceptos y definiciones podemos llegar a la definición final del tipo de la extensión de un record:

\begin{code}
consRec : DecEq lty => {ts : LabelList lty} ->
  {t : Type} -> (l : lty) -> (val : t) ->
  Record ts -> {notElem : Not (ElemLabel l ts)} ->
  Record ((l, t) :: ts)
\end{code}

Donde \texttt{DecEq lty} indica que la etiqueta debe tener igualdad (para la mayoría de los casos basta con \texttt{String}), \texttt{ts : LabelList lty} es la lista de campos del record actual, \texttt{t} es el tipo del nuevo campo, \texttt{l} es la nueva etiqueta, \texttt{val} es el valor del nuevo campo, \texttt{Record ts} es el tipo del record a extender, \texttt{Not (ElemLabel l ts)} la prueba de que \texttt{l} no se encuentra repetida en \texttt{ts}, y el record resultante va a tener el tipo \texttt{Record ((l, t) :: ts)} con el nuevo campo.

\subsection{Listas sin repetidos}

En la sección anterior vimos que para garantizar que no hayan etiquetas repetidas, era necesario tener una prueba de \texttt{Not (ElemLabel l ts)} utilizando los nuevos tipos y funciones definidos anteriormente. Sin embargo, existe otra forma más sencilla de definir tal predicado. En vez de indicar que \texttt{l} no debe pertenecer a \texttt{ls}, se puede probar que la lista \texttt{(l, A) :: ts} no tiene repetidos.

Esto se define con el siguiente predicado:

\begin{code}
data IsSet : List t -> Type where
  IsSetNil : IsSet []
  IsSetCons : Not (Elem x xs) -> IsSet xs ->
    IsSet (x :: xs)
\end{code}

El tipo \texttt{IsSet ls} funciona efectivamente como un predicado lógico, el cual indica que la lista \texttt{ls} es un conjunto que no tiene elementos repetidos.

Las pruebas de este predicado se contruyen de forma inductiva. Primero se prueba que la lista vacía no contiene repetidos. Luego, para el caso inductivo, si se agrega un elemento a una lista, la lista resultante no va a tener repetidos solamente si el elemento a agregar no se encuentra en la lista original.

Como en este trabajo se manejan listas de campos, se define el siguiente tipo:

\begin{code}
IsLabelSet : LabelList lty -> Type
IsLabelSet ts = IsSet (labelsOf ts)
\end{code}

De esta forma, \texttt{consRec} pasa a tener el siguiente tipo:

\begin{code}
consRec : DecEq lty => {ts : LabelList lty} ->
  {t : Type} -> (l : lty) -> (val : t) ->
  Record ts -> {isSet : IsLabelSet ((l, t) :: ts)} ->
  Record ((l, t) :: ts)
\end{code}

\subsection{Construcción de términos de prueba}

El principal problema que ocurre al utilizar la función \texttt{consRec} definida anteriormente se da por el parámetro \texttt{isSet : IsLabelSet ((l, t) :: ts)}. Para poder llamar a esta función es necesario construir una prueba de que la nueva etiqueta a agregar al record no esté repetida en el record ya existente.

Una primera opción es generar la prueba manualmente, pero esto resulta tedioso e impráctico, ya que sería necesario construir ese término de prueba cada vez que se llame a la función.

Una segunda opción es utilizar la decidibilidad del predicado a instanciar. Al tener un tipo decidible, es posible utilizar el typechecker para forzar la unificación del tipo decidible con el tipo mismo o su negación. Básicamente, si uno puede generar un valor del tipo \texttt{Dec p}, entonces puede unificarlo con \texttt{p} en tiempo de compilación, o con \texttt{Not p}. Si en tiempo de compilación la unificación falla, entonces el typechecker va a fallar también.

En este caso en particular, sería necesario poder obtener un valor del tipo \texttt{Dec (IsLabelSet ((l, t) :: ts)}.

Para forzar la unificación, se utilizan estas funciones auxiliares:

\begin{code}
getYes : (d : Dec p) ->
  case d of { No _ => (): Yes _ => p }
getYes (No _) = ()
getYes (Yes yes) = yes

getNo : (d : Dec p) ->
  case d of { No _ => Not p; Yes _ => () }
getNo (No no) = no
getNo (Yes _) = ()
\end{code}

\texttt{getYes} toma un tipo \texttt{Dec p} y retorna una computación la cual hace \textit{pattern matching} sobre el valor de \texttt{Dec p}. Esta computación es ejecutada en tiempo de compilación, y tiene dos opciones: o retorna el tipo \textit{top} \texttt{()}, o retorna el tipo \texttt{p}.

En tiempo de typechecking se realiza pattern matching sobre \texttt{Dec p}. Si la prueba es de \texttt{No}, entonces se retorna el valor \texttt{()} (que efectivamente es un valor del tipo \texttt{()}). Sin embargo, si la prueba es de \texttt{Yes}, entonces ya se tiene un valor de tipo \texttt{p}, por lo cual se retorna ese. El pattern matching no solo permite tener una bifurcación sobre cuál valor retornar, sino también sobre cuál es el tipo de este valor retornado, pudiendo retornar dos valores con tipos totalmente distintos.

La función \texttt{getNo} es idéntica a \texttt{getYes}, pero retorna un valor de tipo \texttt{Not p} en vez de \texttt{p}.

A continuación se muestran ejemplos del uso de estas funciones:

\begin{code}
okYes : Elem "L1" ["L1"]
okYes = getYes $ isElem "L1" ["L1"]
\end{code}

En el caso de \texttt{okYes}, la función \texttt{isElem} es computada en tiempo de compilación, retornando la prueba de \texttt{Elem ''L1'' [''L1'']}. Luego \texttt{getYes} hace pattern matching sobre tal valor y encuentra que se corresponde al caso de \texttt{Yes}, por lo cual retorna ese mismo valor de tipo \texttt{Elem ''L1'' [''L1'']}.

\begin{code}
-- No compila
badYes : Elem "L1" ["L2"]
badYes = getYes $ isElem "L1" ["L2"]
\end{code}

No ocurre lo mismo para el caso de \texttt{badYes}, donde en tiempo de compilación se retorna la prueba de \texttt{Not (Elem ''L1'' [''L2''])}. Al realizar pattern matching entonces \texttt{getYes} retorna \texttt{()}, lo cual no puede ser unificado con \texttt{Elem ''L1'' [''L2'']}, mostrando error de compilación.

\begin{code}
okNo : Not (Elem "L1" ["L2"])
okNo = getNo $ isElem "L1" ["L2"]

-- No compila
badNo : Not (Elem "L1" ["L1"])
badNo = getNo $ isElem "L1" ["L1"]
\end{code}

Lo mismo ocurre de forma inversa con \texttt{okNo} y \texttt{badNo}.

Con este mecanismo se puede generar una prueba automática de cualquier predicado decidible, por lo que se puede simplificar el uso de \texttt{consRec}. Sin embargo, para ello necesitamos probar que \texttt{IsLabelSet} es un tipo decidible, y necesitamos tener una función \texttt{isLabelSet} que permita obtener un valor de tipo \texttt{Dec (IsLabelSet ts)}. Para ello primero se define la función de decidibilidad para \texttt{IsSet}. Esta función, llamada \texttt{isSet}, es la siguiente:

\begin{code}
isSet : DecEq t => (xs : List t) -> Dec (IsSet xs)
isSet [] = Yes IsSetNil
isSet (x :: xs) with (isSet xs)
  isSet (x :: xs) | No notXsIsSet =
    No $ ifNotSetHereThenNeitherThere notXsIsSet
  isSet (x :: xs) | Yes xsIsSet with (isElem x xs)
    isSet (x :: xs) | Yes xsIsSet | No notXInXs =
      Yes $ IsSetCons notXInXs xsIsSet
    isSet (x :: xs) | Yes xsIsSet | Yes xInXs = 
      No $ ifIsElemThenConsIsNotSet xInXs
\end{code}

La función \texttt{isSet} toma una lista de valores que pueden chequearse por igualdad y retorna o una prueba de que no tiene repetidos, o una prueba de que sí los hay. La implementación de \texttt{isSet} realiza un análisis de casos sobre el largo de la lista. Para el caso de la lista vacía ésta no tiene elementos repetidos por definición. Para el caso de que tenga un elemento seguido de la cola de la lista, realiza dos análisis de casos seguidos, verificando si la cola de la lista no tiene repetidos (utilizando recursión), y luego verificando que la cabeza de la lista no pertenezca a la cola de ésta.

En Idris un análisis de casos que tiene impacto en los tipos utiliza el identificador \texttt{with}. Su sintaxis es:

\begin{code}
func params with (expresion)
  func params | Caso1 val1 = ...
  func params | Caso2 val2 = ...
\end{code}

La expresión dentro del \texttt{with} es dividida en sus constructores correspondientes realizando pattern matching, de forma similar a \texttt{case}. La diferencia con \texttt{case} es que \texttt{with} permite redefinir los parámetros anteriores según el resultado del matcheo de la expresión. Al ser Idris un lenguaje con tipos dependientes, pueden ocurrir situaciones donde una expresión matchea solamente cuando otros valores previos de la definición tienen valores específicos. Por ejemplo:

\begin{code}
eq : (n : Nat) -> (m : Nat) -> Bool
eq n m with (decEq n m)
  eq n n | Yes Refl = True
  eq n m | No notNEqM = False
\end{code}

La función \texttt{eq n m} retorna true si \texttt{n} y \texttt{m} son naturales idénticos.

Esta función también introduce el término \texttt{Refl}, el cual es proporcionado por el lenguaje y es equivalente al siguiente constructor:

\begin{code}
data (=) : (a : Type) -> (b : Type) -> Type where
  Refl : x = x
\end{code}

En el caso de \texttt{eq}, el matcheo de \texttt{Yes} tiene un valor de tipo \texttt{val : n = m}. Como la única forma de que se tenga una prueba de ambos es que \texttt{n} sea efectivamente igual a \texttt{m}, se puede unificar \texttt{val} con \texttt{Refl : n = n} y cambiar \texttt{eq n m} por \texttt{eq n n} en la definición de la función.

\begin{code}
ifNotSetHereThenNeitherThere : Not (IsSet xs) ->
  Not (IsSet (x :: xs))
ifNotSetHereThenNeitherThere notXsIsSet
  (IsSetCons xsInXs xsIsSet) = notXsIsSet xsIsSet

ifIsElemThenConsIsNotSet : Elem x xs ->
  Not (IsSet (x :: xs))
ifIsElemThenConsIsNotSet xIsInXs
  (IsSetCons notXIsInXs xsIsSet) = notXIsInXs xIsInXs
\end{code}

Las funciones \texttt{ifNotSetHereThenNeitherThere} y \texttt{ifIsElemThenConsIsNotSet} son dos lemas necesarios para poder definir \texttt{isSet}.

\begin{code}
isLabelSet : DecEq lty => (ts : LabelList lty) ->
  Dec (IsLabelSet ts)
isLabelSet ts = isSet (labelsOf ts)
\end{code}

La función \texttt{isLabelSet} simplemente permite aplicar la función \texttt{isSet} al tipo \texttt{IsLabelSet}.

Teniendo la función \texttt{isLabelSet} es posible generar la prueba para ser utilizada en \texttt{consRec}. Un ejemplo sería el siguiente:

\begin{code}
extendedRec : Record [("Nombre", String)]
extendedRec = consRec "Nombre" "Juan"
  {isSet=(getYes $ isLabelSet ["Nombre"])}
  emptyRec
\end{code}

\subsection{Generación automática de pruebas}

Al utilizar \texttt{getYes} y \texttt{getNo} se simplifica bastante el proceso de construcción de pruebas, pero de todas formas se necesita llamar a esas funciones manualmente. Es posible mejorar este sistema.

A continuación se muestra una técnica alternativa que utiliza el mismo concepto del anterior, donde se realiza pattern matching sobre un tipo decidible en tiempo de compilación para unificar tipos. Sin embargo, el pattern matching se realiza en el tipo mismo y no en una función auxiliar.

Esto es posible gracias a este tipo y esta función:

\begin{code}
TypeOrUnit : Dec p -> Type -> Type
TypeOrUnit (Yes yes) res = res
TypeOrUnit (No _) _ = ()

mkTypeOrUnit : (d : Dec p) -> (cnst : p -> res) ->
  TypeOrUnit d res
mkTypeOrUnit (Yes prf) cnst = cnst prf
mkTypeOrUnit (No _) _ = ()
\end{code}

El tipo \texttt{TypeOrUnit} permite discriminar un tipo en dos casos:
\begin{itemize}
\item Si \texttt{Dec p} incluye una prueba de \texttt{p}, entonces se obtiene el tipo deseado
\item Si \texttt{Dec p} incluye una contradicción de \texttt{p}, entonces se obtiene el tipo \texttt{()}
\end{itemize}

Este método funciona cuando se necesita unificar un tipo \texttt{type} y \texttt{TypeOrUnit dec type}. Si se tiene una prueba de \texttt{p} entonces la unificación va a dar correcta, pero si no se tiene una prueba entonces va a fallar en tiempo de compilación.

\texttt{mkTypeOrUnit} es el constructor de este tipo. Necesita la prueba o contradicción de \texttt{p} y una función que construya el tipo deseado dada una prueba de \texttt{p}. Este constructor es utilizado solamente cuando ya se tiene tal prueba.

Como ejemplo, se tiene una función \texttt{addNat} que debe agregar un natural a una lista solamente si ya pertenece a ésta, y en caso contrario no debe compilar.

\begin{code}
addNat : (n : Nat) -> (ns : List Nat) ->
  TypeOrUnit (isElem n ns) (List Nat)
addNat n ns = mkTypeOrUnit (isElem n ns)
  (\isElem => n :: ns)
\end{code}

La función \texttt{addNat} hace uso de \texttt{TypeOrUnit}, forzando a que se cumpla el predicado \texttt{Elem n ns}. Si no se cumple tal predicado, la función retorna \texttt{()}.

\begin{code}
myListOk : List Nat
myListOk = addNat 10 [10] -- [10, 10]
\end{code}

La llamada a \texttt{addNat 10 [10]} retorna el tipo \texttt{TypeOrUnit (isElem 10 [10]) (List Nat)}, pero \texttt{myListOk} espera \texttt{List Nat}. En tiempo de compilación se evalúa \texttt{isElem 10 [10]}, el cual retorna una prueba de \texttt{Elem 10 [10]}, por lo que \texttt{TypeOrUnit (isElem 10 [10]) (List Nat)} evalúa a \texttt{List Nat}, compilando correctamente el código.

\begin{code}
myListBad1 : List Nat
myListBad1 = addNat 9 [10] -- Error de typechecking
\end{code}

En el caso de \texttt{myListBad1}, como \texttt{isElem 9 [10]} retorna una contradicción de \texttt{Elem 9 [10]}, \texttt{TypeOrUnit (isElem 9 [10]) (List Nat)} evalúa a \texttt{()}, el cual no puede ser unificado con \texttt{List Nat}, generándose un error en tiempo de compilación.

\begin{code}
myListBad2 : Nat -> List Nat
myListBad2 n = addNat n [10] -- Error de typechecking
\end{code}

Otro caso de error ocurre con \texttt{myListBad2}. En este caso es imposible evaluar completamente \texttt{isElem n [10]} ya que no se conoce el valor de \texttt{n} en tiempo de compilación. Por lo tanto no se puede evaluar \texttt{TypeOrUnit (isElem n [10]) (List Nat)}, lo cual hace que falle la unificación con \texttt{List Nat}.

Como conclusión, con este método es posible forzar, en tiempo de compilación, a que se cumpla un predicado específico. Si la función decidible puede ser evaluada hasta su forma normal y retorna \texttt{Yes}, entonces el código compila correctamente. Si la función decidible no puede ser evaluada hasta su forma normal, o puede ser evaluada pero retorna \texttt{No}, entonces el código no compila.

Para poder aplicar este método a los records, es necesario utilizar la función \texttt{isLabelSet} definida en la sección anterior de la siguiente forma:

\begin{code}
RecordOrUnit : DecEq lty => LabelList lty -> Type
RecordOrUnit ts = TypeOrUnit (isLabelSet ts) (Record ts)
\end{code}

\texttt{RecordOrUnit ts} evalúa a \texttt{Record ts} cuando se cumple que \texttt{ts} no tiene repetidos, pero evalúa a \texttt{()} cuando ésta sí tiene repetidos.

Con este tipo es posible tener la siguiente función que extiende un record:

\begin{code}
consRecAuto : DecEq lty => {ts : LabelList lty} ->
  {t : Type} -> (l : lty) -> (val : t) -> Record ts ->
  RecordOrUnit ((l, t) :: ts)
consRecAuto {ts} {t} l val (MkRecord _ hs) =
  mkTypeOrUnit (isLabelSet ((l, t) :: ts))
  (\isLabelSet => MkRecord isLabelSet (val :: hs))
\end{code}

Esta función es idéntica a \texttt{consRec}, solamente que no es necesario pasar una prueba de \texttt{IsLabelSet ((l, t) :: ts)}. Ahora esta prueba se calcula automáticamente en tiempo de compilación y se impacta en el tipo resultante \texttt{RecordOrUnit ((l, t) :: ts)}. Esta implementación se explica a fondo más adelante.

Su uso fue demostrado al comienzo de este capítulo, con el siguiente caso:

\begin{code}
persona : Record [("Clave", Nat), ("Nombre", String),
  ("Edad", Nat)]
persona = consRecAuto "Clave" 3 $
  consRecAuto "Nombre" "Juan" $
  consRecAuto "Edad" 27 $
  emptyRec
\end{code}

Con estas definiciones y técnicas es posible definir el record de arriba, sabiendo en tiempo de compilación que ninguna de las etiquetas utilizadas se repite.

En el caso de que las etiquetas sí se repitan, se mostrará el siguiente mensaje de error:

\begin{code}
recordA : Record [("A", Nat), ("A", Nat)]
recordA = consRecAuto "A" 10 $
  consRecAuto "A" 10 $
  emptyRec

When checking right hand side of recordA with expected type
    Record [("A", Nat), ("A", Nat)]
  Type mismatch between
    RecordOrUnit [("A", Nat), ("A", Nat)]
      (Type of consRecAuto "A" 10
        (consRecAuto "A" 10 emptyRec))
  and
    Record [("A", Nat), ("A", Nat)] (Expected type)
\end{code}

Como \texttt{RecordOrUnit} evalúa a \texttt{()}, nunca puede unificarlo con \texttt{Record}.

\section{Definición de un record}

La implementación del tipo \texttt{Record} de este trabajo es la siguiente:

\begin{code}
data Record : LabelList lty -> Type where
  MkRecord : IsLabelSet ts -> HList ts -> Record ts
\end{code}

Esta definición se corresponde a la de HList de Haskell. Un record es una lista heterogénea donde sus etiquetas no tienen valores repetidos.

Con esta definición se puede ver cómo se implementa \texttt{emptyRec} y \texttt{consRecAuto}.

\begin{code}
emptyRec : Record []
emptyRec = MkRecord IsSetNi {ts=[]} []
\end{code}

Un record vacío simplemente tiene una lista heterogénea vacía y la prueba del caso base de etiquetas no repetidas.

\begin{code}
consRecAuto : DecEq lty => {ts : LabelList lty} ->
  {t : Type} -> (l : lty) -> (val : t) -> Record ts ->
  RecordOrUnit ((l, t) :: ts)
consRecAuto {ts} {t} l val (MkRecord _ hs) =
  mkTypeOrUnit (isLabelSet ((l, t) ::hs))
  (\isLabelSet => MkRecord isLabelSet (val :: hs))
\end{code}

La extensión de un record genera la prueba de que la lista resultante no tiene etiquetas repetidas, y crea el nuevo record con esa prueba y la lista heterogénea extendida con el nuevo elemento.

\subsection{HList actualizado}

En este trabajo se decidió implementar las listas heterogéneas de una forma distinta a la que utiliza Idris en general. Las listas heterogéneas de Idris permiten incluir cualquier tipo arbitrario, pero eso no es suficiente al momento de poder implementar records extensibles. Además es necesario asociar una etiqueta a cada uno de esos tipos.

Se decidió utilizar la siguiente solución, en donde \texttt{HList} no solo tiene el tipo en su lista, sino también la etiqueta:

\begin{code}
data HList : LabelList lty -> Type where
  Nil : HList []
  (::) : {l : lty} -> (val : t) -> HList ts ->
    HList ((l, t) :: ts)
\end{code}

La implementación es idéntica a la de Idris, con la diferencia de que se debe pasar no solo el valor sino la etiqueta también. La etiqueta se guarda solamente en el tipo, y no en la lista en sí.

Un ejemplo sería el siguiente:

\begin{code}
[1, "Juan"] :
  HList [("Clave", Nat), ("Nombre", String)]
\end{code}

\section{Implementación de operaciones sobre records}

Al comienzo de esta sección vimos algunas operaciones sobre records y su uso. Ahora describiremos cómo se implementan.

\subsection{Proyección sobre un record}

El ejemplo visto anteriormente fue el siguiente:

\begin{code}
personaYDireccion : Record [("Clave", Nat), ("Nombre", String),
  ("Edad", Nat), ("Direccion", String)]

proyeccion : Record [("Clave", Nat), ("Direccion", String)]
proyeccion = hProjectByLabelsAuto ["Clave", "Direccion"]
  personaYDireccion
\end{code}

La proyección sobre un record toma una lista de etiquetas y retorna solo los campos del record asociados a esas etiquetas, es decir, realiza una proyección del record.

Esta función utiliza la misma técnica de \texttt{consRecAuto}. Las etiquetas que se pasan en la lista no tienen que estar repetidas, por lo que tal propiedad debe ser verificada en tiempo de compilación. 

\begin{code}
proyeccion : Record [("Clave", Nat), ("Clave", Nat)]
proyeccion = HProjectByLabelsAuto ["Clave", "Clave"]
  personaYDireccion
-- No compila
\end{code}

Tales etiquetas pueden o no estar en el record; si no están en el record simplemente se ignoran. El tipo de \texttt{hProjectByLabelsAuto} es el siguiente:

\begin{code}
hProjectByLabelsAuto : DecEq lty => {ts : LabelList lty} ->
  (ls : List lty) -> Record ts ->
  TypeOrUnit (isSet ls) (Record (projectLeft ls ts))
\end{code}

Esta función toma una lista de etiquetas, un record, y automáticamente genera una prueba de \texttt{IsSet ls}. En caso de poder hacerlo, retorna un nuevo record \texttt{Record (projectLeft ls ts)}, donde \texttt{projectLeft} es una función a nivel de tipos que permite retornar una lista a nivel de tipos distinta según los campos a proyectar.

\begin{code}
projectLeft ["Clave"
  [("Clave", Nat), ("Nombre", String)] =
  [("Clave", Nat)]

projectLeft ["Clave", "Direccion"]
  [("Clave", Nat), ("Nombre", String), ("Direccion", String)] =
  [("Clave", Nat), ("Direccion", String)]
\end{code}

Esta es una función que se puede usar sobre listas, pero puede ser también utilizada para tener distintos tipos de un record. La computación simplemente se translada a los tipos.

\begin{code}
Record (projectLeft ["Clave", "Direccion"]
  [("Clave", Nat), ("Nombre", String), ("Direccion", String)]) =
Record [("Clave", Nat), ("Direccion", String)]
\end{code}

La implementación de \texttt{projectLeft} es la siguiente:

\begin{code}
projectLeft : DecEq lty => List lty -> LabelList lty ->
  LabelList lty
projectLeft [] ts = []
projectLeft ls [] = []
projectLeft ls ((l, ty) :: ts) with (isElem l ls)
  projectLeft ls ((l, ty) :: ts) | Yes lIsInLs =
    let delLFromLs = deleteElem ls lIsInLs
      rest = projectLeft delLFromLs ts
    in (l, ty) :: rest
  projectLeft ls ((l, ty) :: ts) | No _ = projectLeft ls ts

deleteElem : (xs : List t) -> Elem x xs -> List t
deleteElem (x :: xs) Here = xs
deleteElem (x :: xs) (There inThere) =
  x :: (deleteElem xs inThere)
\end{code}

La implementación hace recursión sobre los campos del record. Si el primer campo pertenece a la lista a proyectar, elimina ese campo de la lista a proyectar y retorna el campo elegido, aplicando el mismo razonamiento para el resto de la lista. Si no pertence, no retorna ese campo y aplica la función al resto de la lista.

Con esto se puede implementar \texttt{hProjectByLabelsAuto}

\begin{code}
hProjectByLabelsAuto : DecEq lty => {ts : LabelList lty} ->
  (ls : List lty) -> Record ts ->
  TypeOrUnit (isSet ls) (Record (projectLeft ls ts))
hProjectByLabelsAuto {ts} ls rec =
  mkTypeOrUnit (isSet ls) (\lsIsSet =>
    hProjectByLabels {ts} ls rec lsIsSet)
\end{code}

Simplemente aplica la misma técnica de \texttt{consRecAuto} con \texttt{TypeOrUnit}, delegando la llamada a \texttt{hProjectByLabels}, cuya implementación es la siguiente:

\begin{code}
hProjectByLabels : DecEq lty => {ts : LabelList lty} ->
  (ls : List lty) -> Record ts -> IsSet ls ->
  Record (projectLeft ls ts)
hProjectByLabels {ts} ls rec lsIsSet =
  let isLabelSet = recLblIsSet rec
      hs = recToHList rec
      (lsRes ** (hsRes, prjLeftRes)) =
        fst $ hProjectByLabelsHList ls hs
      isLabelSetRes =
        hProjectByLabelsLeftIsSet_Lemma2 prjLeftRes isLabelSet
      resIsProjComp = fromIsProjectLeftToComp prjLeftRes lsIsSet
      recRes = hListToRec {prf=isLabelSetRes} hsRes
  in rewrite (sym resIsProjComp) in recRes
\end{code}

Para entender esta implementación iremos por partes. Comencemos por:

\begin{code}
isLabelSet = recLblIsSet rec
hs = recToHList rec
\end{code}

Dado un record, se obtiene su lista heterogénea y la prueba de que no tiene etiquetas repetidas, utilizando las siguientes funciones:

\begin{code}
recToHList : Record ts -> HList ts
recToHList (MkRecord _ hs) = hs

recLblIsSet : Record ts -> IsLabelSet ts
recLblIsSet (MkRecord lsIsSet _) = lsIsSet
\end{code}

\begin{code}
(lsRes ** (hsRes, prjLeftRes)) =
  fst $ hProjectByLabelsHList ls hs
\end{code}

Esta llamada hace uso de la siguiente función:

\begin{code}
hProjectByLabelsHList : DecEq lty => {ts : LabelList lty} ->
  (ls : List lty) -> HList ts ->
  ((ls1 : LabelList lty ** (HList ls1, IsProjectLeft ls ts ls1)),
  (ls2 : LabelList lty ** (HList ls2, IsProjectRight ls ts ls2)))
\end{code}

Donde la sintaxis \texttt{(x : a ** (P a))} indica que se tiene un par dependiente, donde el tipo del segundo valor depende del primero. Si se tiene un tipo \texttt{A} y otro tipo \texttt{B}, se puede tener un par \texttt{(A, B)}. Sin embargo, si \texttt{B} depende de un valor de \texttt{A}, se puede tener un par dependiente con \texttt{(a : A ** (B a))}.

La función anterior realiza la proyección a nivel de listas heterogéneas. También contiene otro mecanismo de manejo de tipos, donde en vez de retornar la computación en el tipo mismo, retorna un predicado que representa esa computación. El tipo \texttt{IsProjectLeft} es el tipo que cumple la siguiente propiedad:

\begin{code}
IsProjectLeft ls ts1 ts2 <-> ts2 = projectLeft ls ts1
\end{code}

Representa la proposición \textit{''Si se proyecta 'ls' sobre 'ts1', el resultado es 'ts2' ''}. \texttt{IsProjectRight} cumple el mismo propósito, pero realiza la proyección por derecha (retorna todos los elementos que no fueron proyectados por \texttt{projectLeft}).

Su definición es la siguiente:

\begin{code}
data IsProjectLeft : DecEq lty => List lty -> LabelList lty ->
  LabelList lty -> Type where
  IPL_EmptyLabels : DecEq lty => IsProjectLeft {lty} [] ts []
  IPL_EmptyVect : DecEq lty => IsProjectLeft {lty} ls [] []
  IPL_ProjLabelElem : DecEq lty => (isElem : Elem l ls) ->
    DeleteElemPred ls isElem lsNew ->
    IsProjectLeft {lty} lsNew ts res1 ->
    IsProjectLeft ls ((l, ty) :: ts) ((l, ty) :: rs1)
  IPL_ProjLabelNotElem : DecEq lty => Not (Elem l ls) ->
    IsProjectLeft {lty} ls ts res1 ->
    IsProjectLeft ls ((l, ty) :: ts) res1

data DeleteElemPred : (xs : List t) -> Elem x xs ->
  List t -> Type where
  DeleteElemPredHere : DeleteElemPred (x :: xs) Here xs
  DeleteElemPredThere : {isThere : Elem y xs} ->
    DeleteElemPred xs isThere ys ->
    DeleteElemPred (x :: xs) (There isThere) (x :: ys)
\end{code}

Como se ve, la definición de \texttt{IsProjectLeft} es muy similar a la de \texttt{projectLeft}. Esto es necesario para que se ambos sean isomorfismos (descrito por la propiedad anterior). Se pueden ver tales similitudes comparando ambas implementaciones línea a línea:

\begin{code}
IPL_EMptyLabels : DecEq lty => IsProjectLeft {lty} [] ts []
projectLeft [] ts = []

IPL_EmptyVect : DecEq lty => IsProjectLeft {lty} ls [] []
projectLeft ls [] []

IPL_ProjLabelElem : DecEq lty => (isElem : Elem l ls) ->
  DeleteElemPred ls isElem lsNew ->
  IsProjectLeft {lty} lsNew ts res1 ->
  IsProjectLeft ls ((l, ty) :: ts) ((l, ty) :: res1)
projectLeft ls ((l, ty) :: ts) | Yes lIsInLs =
  let delLFromLs = deleteElem ls lIsInLs
    rest = projectLet delLFromLs ts
  in (l, ty) :: rest

IPL_ProjLabelNotElem : DecEq lty => Not (Elem l ls) ->
  IsProjectLeft {lty} ls ts res1 ->
  IsProjectLeft ls ((l, ty) :: ts) res1
projectLeft ls ((l, ty) :: ts) | No _ = projectLeft ls ts
\end{code}

Cada constructor del predicado se corresponde a cada caso de pattern matching de la definición de la función (sea pattern matching regular o utilizando \texttt{with}).

Algunos ejemplos son los siguientes:

\begin{code}
IsProjectLeft ["Edad"] [("Edad", Nat), ("Nombre", String)]
  [("Edad", Nat)]
IsProjectRight ["Edad"] [("Edad", Nat), ("Nombre", String)]
  [("Nombre", String)]
\end{code}

Con estas definiciones, los siguientes son equivalentes:

\begin{code}
(ls1 : LabelList lty ** (Hlist ls1, IsProjectLeft ls ts ls1)

HList (projectLeft ls ts)
\end{code}

En este trabajo se decidió utilizar los predicados para operaciones internas porque resulta más sencillo realizar pattern matching sobre sus constructores que sobre los argumentos de la función original.

Cada vez que se tiene un llamado a una función que retorna \texttt{projectLeft} también es necesario hacer pattern matching sobre los argumentos específicos que la implementación de esa función hace pattern matching, en el orden exacto en que los hace. Sin embargo, con predicados como \texttt{IsProjectLeft} solo basta hacer pattern matching sobre los constructores y nada más. De esta forma el desarrollo queda más sencillo.

El resto de la implementación de \texttt{hProjectByLabelsHList} se puede ver en el apéndice.

Siguiendo con la implementación de \texttt{hProjectLabels}, se tiene lo siguiente:

\begin{code}
isLabelSetRes =
  hProjectByLabelsLeftIsSet_Lemma2 prjLeftRes isLabelSet
\end{code}

Donde el lema es el siguiente:

\begin{code}
hProjectByLabelsLeftIsSet_Lemma2 : DecEq lty => {ls : List lty} ->
  {ts1, ts2 : LabelList lty} -> IsProjectLeft ls ts1 ts2 ->
  IsLabelSet ts1 -> IsLabelSet ts2
\end{code}

Indica que si se hace proyección sobre una lista de campos, y no hay etiquetas repetidas en los campos originales entonces tampoco los van a haber en los campos resultantes. Su implementación se puede encontrar en el apéndice.

\begin{code}
resIsProjComp = fromIsProjectLeftToComp prfLeftRes lsIsSet

fromIsProjectLeftToComp : DecEq lty => {ls : List lty} ->
  {ts1, ts2 : LabelList lty} -> IsProjectLeft ls ts1 ts2 ->
  IsSet ls -> ts2 = projectLeft ls ts1
\end{code}

Donde la función representa la propiedad que debe cumplir \texttt{IsProjectLeft} en relación a \texttt{projectLeft} que vimos anteriormente. Se le agrega el hecho de que \texttt{ls} no debe tener elementos repetidos, que es necesario para que esta función pueda ser implementada. Su implementación también está en el apéndice.

\begin{code}
recRes = hListToRec {prf=isLabelSetRes} hsRes
\end{code}

La función \texttt{hListToRec} construye un record a partir de una lista heterogénea y una prueba de etiquetas no repetidas:

\begin{code}
hListToRec : DecEq lty => {ts : LabelList lty} ->
  {prf : IsLabelSet ts} -> HList ts -> Record ts
hListToRec {prf} hs = MkRecord prf hs
\end{code}

Por último se tiene esta línea de código:

\begin{code}
in rewrite (sym resIsProjComp) in recRes
\end{code}

En Idris este es un caso de reescribir un término teniendo una prueba de igualdad. Por la llamada a \texttt{hProjectByLabelsHList} y \texttt{hListToRec} se tiene un término \texttt{recRes : Record ts2}. Por la llamada a \texttt{fromIsProjectLeftToComp} se tiene un término \texttt{resIsProjComp : ts2 = projectLeft ls ts1}. La función \texttt{sym} simplemente cambia el orden de una igualdad, por lo tanto \texttt{(sym resIsProjComp) : projectLeft ls ts1 = ts2}. Consecuentemente con esta técnica de reescritura se puede obtener un término del tipo \texttt{Record (projectLeft ls ts1)}, tal como lo indica el tipo de la función.

\subsection{Búsqueda de un elemento en un record}

Al comienzo del capítulo vimos un ejemplo de buscar el valor de un campo en particular de un record:

\begin{code}
persona : Record [("Clave", Nat), ("Nombre", String),
  ("Edad", Nat)]

nombre : String
nombre = hLookupByLabelAuto "Nombre" persona
\end{code}

La funcionalidad de lookup se puede definir de forma muy similar a las anteriores. Primero se comienza con un predicado que indica que una lista de etiquetas contiene a otra etiqueta con un determinado tipo en particular:

\begin{code}
data HasField : (l : lty) -> LabelList lty ->
  Type -> Type where
  HasFieldHere : HasField l ((l, ty) :: ts) ty
  HasFieldTHere : HasField l1 ts ty1 ->
    HasField l1 ((l2, ty2) :: ts) ty1
\end{code}

\texttt{HasField l ts ty} indica que en la lista de etiquetas \texttt{ts} existe la etiqueta \texttt{l} que tiene asociado el tipo \texttt{ty}. Un ejemplo de tal tipo es \texttt{HasField ''Edad'' [(''Nombre'', String), (''Edad'', Nat)] Nat}. Su definición es inductiva, donde \texttt{HasFieldHere} indica que la etiqueta se encuentra en la cabeza de la lista, y \texttt{HasFieldThere} indica que se encuentra en la cola de la lista.

Una primera definición es la de obtener el valor de un campo de una lista heterogénea:

\begin{code}
hLookupByLabel_HList : DecEq lty => {ts : LabelList lty} ->
  (l : lty) -> HList ts -> HasField l ts ty -> ty
hLookupByLabel_HList _ (val :: _) HasFieldHere = val
hLookupByLabel_HList l (_ :: ts) 
  (HasFieldThere hasFieldThere) =
  hLookupByLabel_HList l ts hasFieldThere
\end{code}

Esta función toma una etiqueta de tipo \texttt{lty}, una lista heterogénea de tipo \texttt{HList ts}, y una prueba de que esa etiqueta pertenece a la lista de tipo \texttt{HasField l ts ty}. Con tales datos es posible entonces retornar el valor de tipo \texttt{ty} asociado a tal etiqueta. Su implementación realiza pattern matching sobre el predicado, obteniendo el valor \texttt{val : ty} si la etiqueta está en la cabeza de la lista, o sino lo obtiene realizando un llamado recursivo.

Con esta función se puede definir \texttt{hLookupByLabel} de la siguiente forma:

\begin{code}
hLookupByLabel : DecEq lty => {ts : LabelList lty} ->
  (l : lty) -> Record ts -> HasField l ts ty -> ty
hLookupByLabel {ts} {ty} l rec hasField =
  hLookupByLabel_HList {ts} {ty} l (recToHList rec) hasField
\end{code}

Como un record se define en términos de una lista heterogénea, entonces esta función simplemente aplica la función definida anteriormente a tal lista heterogénea contenida en el record.

Al igual que las funcionalidades anteriores, se puede definir una función que calcule el predicado de forma automática en tiempo de compilación. Sin embargo, con el caso de lookup ocurre un problema. Idealmente, se quisiera tener la siguiente función:

\begin{code}
hasField : DecEq lty => (l : lty) ->
  (ts : LabelList lty) -> (ty : Type) ->
  Dec (HasField l ts ty)
\end{code}

Con esta función se podría definir la función de cálculo automático de la siguiente forma:

\begin{code}
hLookupByLabelAuto : DecEq lty => {ts : LabelList lty} ->
  (l : lty) -> Record ts ->
  TypeOrUnit (hasField l ts ty) ty
hLookupByLabelAuto {ts} {ty} l rec =
  mkTypeOrUnit (hasField l ts ty)
  (\tsHasL => hLookupByLabel {ts} {ty} l rec tsHasL)
\end{code}

El problema es que no es posible definir \texttt{hasField}. Para implementarla, se tiene una variable \texttt{ty : Type} que puede pertenecer a cualquier tipo. A su vez, si la etiqueta \texttt{l : lty} pertenece a la lista, entonces en la lista la etiqueta va a tener asociado un tipo \texttt{ty2 : Type} que puede ser cualquier otro. El problema ocurre en que no se puede verificar la igualdad de los tipos \texttt{ty} y \texttt{ty2}, ya que no existe forma de igualar valores de tipo \texttt{Type}.

Por ejemplo, se podría tener \texttt{hasField ''Edad'' [(''Edad'', Nat)] Nat} o \texttt{hasField ''Edad'' [(''Edad'', Nat)] String}. El primer caso debería retornar \texttt{Yes} con una prueba, mientras que el segundo deberia retornar \texttt{No} con una contradicción, ya que el tipo \texttt{Nat} y \texttt{String} son distintos. Sin embargo, no es posible realizar un chequeo \texttt{Nat = Nat} o \texttt{Not (Nat = String)}, ya que eso requeriría que existiese una instancia de \texttt{DecEq Type}, la cual no existe y no es posible definir.

El hecho de que no existe una instancia \texttt{DecEq Type} en el lenguaje tiene varias razones:

\begin{itemize}
\item Dados dos tipos cualesquiera en un lenguaje de tipos dependientes, verificar si son iguales no es decidible. Es decir, teóricamente, es imposible crear una instancia de \texttt{DecEq Type}.
\item Si existiera tal forma, rompería el polimorfismo paramétrico. El polimorfismo paramétrico permite abstraerse de un tipo y trabajar con él sin saber específicamente cuál es. Esto permite deducir propiedades de una función solamente conociendo su tipo (esto se conoce como propiedad de parametricidad o free theorems \cite{Wadler:1989:TF:99370.99404}). Por ejemplo, si se tiene la función \texttt{id : a -> a}, se puede deducir que la única posible implementación es \texttt{id val = val}, ya que es imposible que la función pueda conocer información sobre el tipo \texttt{a : Type}.

Sin embargo, si existiera una instancia de \texttt{DecEq Type}, entonces la siguiente función sería válida:

\begin{code}
id2 : a -> a
id2 {a} val with (decEq a Nat)
  id2 {a=Nat} val | Yes _ = 10
  id2 {a} val | No _ = val
\end{code}

La función \texttt{id2 : a -> a} retorna el mismo valor para todos los tipos, menos para el tipo \texttt{Nat}, donde retorna siempre el valor 10. Esta función contradice las garantías de parametricidad del tipo \texttt{a -> a}.
\end{itemize}

A pesar de no poder utilizar \texttt{TypeOrUnit}, es posible definir una función que calcule el predicado de forma automática, utilizando la funcionalidad \texttt{auto} de Idris, de esta forma:

\begin{code}
hLookupByLabelAuto : DecEq lty => {ts : LabelList lty} ->
  (l : lty) -> Record ts ->
  {auto hasField : HasField l ts ty} -> ty
hLookupByLabelAuto {ts} {ty} l rec {hasField} =
  hLookupByLabel_HList {ts} {ty} l (recToHList rec) hasField
\end{code}

La expresión \texttt{auto hasField : HasField l ts ty} indica que el typechecker de Idris va a intentar generar una prueba del predicado \texttt{HasField l ts ty} de forma automática. Idris se da cuenta de que la definición de \texttt{HasField} es inductiva y tiene distintos constructores, por lo que intenta, mediante fuerza bruta, aplicar los constructores secuencialmente hasta encontrar un valor que tenga el tipo deseado. Este mecanismo se lo denominia \textit{táctica}.

Por ejemplo, si se quisiera obtener una prueba de \texttt{HasField ''Edad'' [(''Nombre'', String), (''Edad'', Nat)] Nat}, Idris primero va a construir \texttt{HasFieldHere}. Como \texttt{HasFieldHere} no puede unificarse con el tipo deseado, sigue con el siguiente caso. Próximamente, Idris intenta construir \texttt{HasFieldThere HasFieldHere}. Idris consigue en ese caso unificar el tipo de ese valor con el esperado y entonces proporciona el valor \texttt{HasFieldThere HasFieldHere} a la función \texttt{hLookupByLabelAuto}.

Como conclusión, utilizar \texttt{TypeOrUnit} es muy útil para definir estas funciones con un cálculo automático de los predicados correspondientes, pero solo puede usarse si el predicado no depende de un valor \texttt{a : Type} (o de cualquier valor que no puede verificarse por igualdad). Si es así, se deben utilizar funcionalidades propias del lenguaje, como \texttt{auto}.

\subsection{Unión por izquierda}

Al comienzo del capítulo se vió el siguiente ejemplo de unión por izquierda de records:

\begin{code}
persona : Record [("Clave", Nat), ("Nombre", String),
  ("Edad", Nat)]
persona = hLeftUnion personaConNombre personaConEdad
\end{code}

La función \texttt{hLeftUnion} toma dos records cualesquiera, y retorna uno nuevo con todos los campos del de la izquierda, más los campos del de la derecha que no están repetidos en el de la izquierda. Básicamente realiza una unión de los campos de ambos records, con prioridad por los campos del record de la izquierda en caso de etiquetas repetidas.

Su definición es la siguiente:

\begin{code}
hLeftUnion : DecEq lty => {ts1, ts2 : LabelList lty} ->
  Record ts1 -> Record ts2 -> Record (hLeftUnion_List ts1 ts2)
hLeftUnion ts1 ts2 =
  let (tsRes ** (resUnion, isLeftUnion)) =
      hLeftUnionPred ts1 ts2
    leftUnionEq = fromHLeftUnionPredToFunc isLeftUnion
  in rewrite (sym leftUnionEq) in resUnion
\end{code}

Al obtener un record nuevo se debe computar la lista de sus campos. Esto se hace con la función \texttt{hLeftUnion\_List}:

\begin{code}
hLeftUnion_List : DecEq lty => LabelList lty ->
  LabelList lty -> LabelList lty
hLeftUnion_List ts1 ts2 =
  ts1 ++ (deleteLabels (labelsOf ts1) ts2)

deleteLabels : DecEq lty => List lty -> LabelList lty ->
  LabelList lty
deleteLabels [] ts = ts
deleteLabels (l :: ls) ts =
  deleteLabelAt l (deleteLabels ls ts)

deleteLabelAt : DecEq lty => lty -> LabelList lty ->
  LabelList lty
deleteLabelAt l [] = []
deleteLabelAt l1 ((l2, ty) :: ts) with (decEq l1 l2)
  deleteLabelAt l1 ((l2, ty) :: ts) | Yes _ = ts
  deleteLabelAt l1 ((l2, ty) :: ts) | No _ =
    (l2, ty) :: deleteLabelAt l1 ts
\end{code}

La función \texttt{hLeftUnion\_List} es un poco más compleja que las anteriores. Se define como la unión de dos listas tomando la primera, y agregándole todos los campos de la segunda, menos aquellos que tienen etiquetas que aparecen en la primera lista. Esta eliminación se realiza con \texttt{deleteLabels}, que simplemente recorre toda la lista de campos a eliminar y los elimina de la segunda (utilizando la función \texttt{deleteLabelAt}).

Al igual que en \texttt{hProjectByLabels}, \texttt{hLeftUnion} hace uso de un predicado que representa la computación de la unión de los campos del record, y delega la implementación a una función que retorna ese predicado.

\begin{code}
hLeftUnionPred : DecEq lty => {ts1, ts2 : LabelList lty} ->
  Record ts1 -> Record ts2 ->
  (tsRes : LabelList lty ** (Record tsRes,
    IsLeftUnion ts1 ts2 tsRes))
\end{code}

Al igual que \texttt{IsProjectLeft}, \texttt{IsLeftUnion} es el predicado que cumple esta propiedad:

\begin{code}
IsLeftUnion ts1 ts2 ts3 <-> ts3 = hLeftUnion_List ts1 ts2
\end{code}

Una de estas propiedades es la utilizada en la siguiente función vista más arriba:

\begin{code}
fromHLeftUnionFuncToPred : DecEq lty =>
  {ts1, ts2 : LabelList lty} ->
  IsLeftUnion ts1 ts2 (hLeftUnion_List ts1 ts2)
\end{code}

A continuación se muestra la definición del tipo \texttt{IsLeftUnion}, que al igual que \texttt{IsProjectLeft}, se corresponde uno a uno con la definición de su función análoga (\texttt{hLeftUnion\_List} en este caso):

\begin{code}
data IsLeftUnion : DecEq lty => LabelList lty -> LabelList lty ->
  LabelList lty -> Type where
  IsLeftUnionAppend : DecEq lty =>
    {ts1, ts2, ts2 : LabelList lty} ->
    DeleteLabelsPred (labelsOf ts1) ts2 ts3 ->
    IsLeftUnion ts1 ts2 (ts1 ++ ts3)

data DeleteLabelsPred : DecEq lty => List lty -> LabelList lty ->
  LabelList lty -> Type where
  EmptyLabelList : DecEq lty =>
    DeleteLabelsPred {lty} [] ts ts
  DeleteFirstOfLabelList : DecEq lty =>
    DeleteLabelAtPred l tsAux tsRes ->
    DeleteLabelsPred ls ts tsAux ->
    DeleteLabelsred {lty} (l :: ls) ts tsRes

data DeleteLabelAtPred : DecEq lty -> lty -> LabelList lty ->
  LabelList lty -> Type where
  EmptyRecord : DecEq lty => {l : lty} ->
    DeleteLabelAtPred l [] []
  IsElem : DecEq lty => {l : lty} ->
    DeleteLabelAtPred l ((l, ty) :: ts) ts
  IsNotElem : DecEq lty => {l1 : lty} -> Not (l1 = l2) ->
    DeleteLabelAtPred l1 ts1 ts2 ->
    DeleteLabelAtPred l1 ((l2, ty) :: ts1) ((l2, ty) :: ts2)
\end{code}

El resto de la implementación de \texttt{hLeftUnion} se puede ver en el apéndice.

\section{Ejemplos}

En esta sección se muestra cómo se escriben los ejemplos del capítulo \ref{ch:2} con la solución de records extensibles de este trabajo.

El ejemplo visto para Elm era el siguiente:

\begin{code}
type alias Positioned a =
  { a | x : Float, y : Float }

type alias Named a =
  { a | name : String }

type alias Moving a =
  { a | velocity : Float, angle : Float }

lady : Named { age : Int }
lady =
  { name = "Lois Lane"
  , age = 31
  }

dude : Named (Moving (Positioned {}))
dude =
  { x = 0
  , y = 0
  , name = "Clark Kent"
  , velocity = 42
  , angle = degrees 30
  }
\end{code}

Este código puede ser escrito en Idris de la siguiente forma:

\begin{code}
Positioned : LabelList String -> LabelList String
Positioned ts = ("x", Double) :: ("y", Double) :: ts

Named : LabelList String -> LabelList String
Named ts = ("name", String) :: ts

Moving : LabelList String -> LabelList String
Moving ts = ("velocity", Double) :: ("angle", Double) :: ts

lady : Record (Named [("age", Nat)])
lady = consRecAuto "name" "Lois Lane" $
  consRecAuto "age" 31 $
  emptyRec

dude : Record (Named . Moving . Positioned $ [])
dude = consRecAuto "name" "Clark Kent" $
  consRecAuto "velocity" 42 $
  consRecAuto "angle" 30 $
  consRecAuto "x" 0 $
  consRecAuto "y" 0 $
  emptyRec
\end{code}

Esta es la traducción más directa del ejemplo de Elm. Sin embargo, si se quiere mejorar el diseño, se puede utilizar la flexibilidad de Idris para diseñarlo de otra forma (sea utilizando typeclasses, tipos de datos, etc).

El ejemplo visto para Purescript era el siguiente:

\begin{code}
fullName :: forall t. { firstname :: String,
  lastName :: String | t } -> String
fullName person = person.firstName ++ " " ++ person.lastName
\end{code}

Esto puede traducirse a Idris de la siguiente forma:

\begin{code}
fullName : {ts : LabelList String}
  {auto hasFirst : HasField "firstName" ts String} ->
  {auto hasLast : HasField "lastName" ts String} ->
  Record ts -> String
fullName {hasFirst} {hasLast} rec =
  let firstName = hLookupByLabel "firstName" rec has First
    lastName = hLookupByLabel "lastName" rec hasLast
  in firstName ++ " " ++ lastName
\end{code}

Un ejemplo de su uso sería el siguiente:

\begin{code}
persona : Record [("firstName", String), ("age", Nat),
  ("lastName", String)]
persona = consRecAuto "firstName" "Juan" $ 
  consRecAuto "age" 23 $
  consRecAuto "lastName" "Sanchez" $
  emptyRec

fullName persona : String
-- "Juan Sanchez"
\end{code}

En Idris se puede parametrizar por cualquier predicado, en particular por \texttt{HasField}, permitiendo la programación con \textit{row polymorphism} de forma similar a Purescript y otros lenguajes.

Comparando con Elm y Purescript, al ser todos los resultados del tipo \texttt{Record} se pueden utilizar todas las operaciones vistas hasta ahora en este capítulo (como \texttt{hLeftUnion}, \texttt{hProjectLabels}, etc), lo cual estos lenguajes no soportan.



