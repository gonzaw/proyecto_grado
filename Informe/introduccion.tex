%%%%%%%%%%%%%%%%%%%%%%%%%%%%%%%%%%%%%%%%%%%%%%%%%%
% Records Extensibles - Introducción
%%%%%%%%%%%%%%%%%%%%%%%%%%%%%%%%%%%%%%%%%%%%%%%%%%

\chapter{Introducción}
\label{ch:1}

Los records extensibles son una herramienta muy útil en la programación. Surgen como una respuesta a un problema que tienen los lenguajes de programación con tipado estático en cuanto a records: ¿Cómo puedo modificar la estructura de un record ya definido?

En los lenguajes de programación fuertemente tipados modernos no existe una forma primitiva de definir y manipular records extensibles. Algunos lenguajes ni siquiera permiten definirlos.

Este trabajo se enfoca en presentar una manera de definir records extensibles en lenguajes funcionales fuertemente tipados. En particular, se presenta una manera de hacerlo utilizando un lenguaje de programación con \textit{tipos dependientes} llamado Idris.

\section{Records extensibles}

En varios lenguajes de programación con tipos estáticos, es posible definir una estructura estática llamada 'record'. Un record es una estructura heterogénea que permite agrupar valores de varios tipos en un único objeto, asociando una etiqueta o nombre a cada uno de esos valores.

Un ejemplo de la definición de un tipo record en Haskell sería la siguiente:

\begin{code}
data Persona = Persona { edad :: Int, nombre :: String }
\end{code}

En muchos lenguajes de programación, una vez definidos los records no es posible modificar su estructura de forma dinámica. Tomando el ejemplo anterior, si uno quisiera tener un nuevo record con los mismos campos de \texttt{Persona} pero con un nuevo campo adicional, como \texttt{apellido}, entonces uno solo podría definirlo de una de estas dos formas:

\begin{itemize}[noitemsep]
\item Definir un nuevo record con esos 3 campos.
\begin{code}
data Persona2 = Persona2 {
  edad :: Int,
  nombre :: String,
  apellido :: String
}
\end{code}
\item Crear un nuevo record que contenga el campo \texttt{apellido} y otro campo de tipo \texttt{Persona}.
\begin{code}
data Persona2 = Persona2 {
  datosViejos :: Persona,
  apellido :: String
}
\end{code}
\end{itemize}

En ninguno de ambos enfoques es posible obtener el nuevo record extendiendo el anterior de manera dinámica. Es decir, siempre es necesario definir un nuevo tipo, indicando que es un record e indicando sus campos.

Los records extensibles intentan resolver este problema. Si \texttt{Persona} fuera un record extensible, entonces definir un nuevo record idéntico a él, pero con un campo adicional, sería tan fácil como tomar un record de tipo \texttt{Persona} existente, y simplemente agregarle el nuevo campo \texttt{apellido} con su valor correspondiente. A continuación se muestra un ejemplo de records extensibles, con una sintaxis hipotética similar a Haskell:

\begin{code}
p :: Persona
p = Persona { edad = 20, nombre = "Juan" }

pExtensible :: Persona + { apellido :: String }
pExtensible = p + { apellido = "Sanchez" }
\end{code}

El tipo del nuevo record debería reflejar el hecho de que es una copia de \texttt{Persona} pero con el campo adicional utilizado. Uno debería poder, por ejemplo, acceder al campo \texttt{apellido} del nuevo record como uno lo haría con cualquier otro record arbitrario. A su vez, se pueden acceder a los campos \texttt{edad} y \texttt{nombre} como se podía hacerlo antes de extender el record.

Para poder implementar un record extensible, es necesario poder codificar toda la información del record. Esta información incluye las etiquetas que acepta el record, y el tipo de los valores asociados a cada una de esas etiquetas.

Algunos lenguajes permiten implementar records extensibles con primitivas del lenguaje (como en Elm \cite{ElmRecords} o Purescript \cite{PurescriptRecords}), pero requieren de sintaxis y semántica especial para los records extensibles, y no es posible implementarlos en términos de primitivas del lenguaje más básicas.

Las propiedades de records extensibles requieren que los campos sean variables y puedan ser agregados a records previamente definidos. En el ejemplo visto más arriba, básicamente se espera poder tener un tipo \texttt{Persona}, y poder agregarle un campo \texttt{apellido} con el tipo \texttt{String}, y que esto sea un nuevo tipo. Esto se puede realizar con \textit{tipos dependientes}.

\section{Tipos dependientes}

\textit{Tipos dependientes} son tipos que dependen de valores, y no solamente de otros tipos. En el ejemplo anterior, el nuevo tipo \texttt{Persona + \{ apellido :: String \}} depende de tipos (\texttt{Persona} y \texttt{String}) pero también de valores. El elemento \texttt{apellido} se puede representar, por ejemplo, como un valor de tipo \texttt{String}, quedando el tipo final \texttt{Persona + \{ 'apellido' :: String \}}. Al ser un valor, se podría, por ejemplo, guardarlo en una variable y utilizarlo de la siguiente manera:

\begin{code}
miCampo = String
miCampo = 'apellido'

pExtensible :: Persona + { miCampo :: String }
\end{code}

Con tipos dependientes los tipos son \textit{first-class}, lo que permite que cualquier operación que se puede realizar sobre valores también se puede realizar sobre tipos. En particular, se pueden definir funciones que tengan tipos como parámetros y retornen tipos como resultado. En el ejemplo, \texttt{(+)} funciona como tal función, tomando el tipo \texttt{Persona} y el tipo \texttt{\{ 'apellido' :: String \}} y uniéndolos en un nuevo tipo. \texttt{(::)} funciona como otra función, uniendo el valor \texttt{apellido} y el tipo \texttt{String}.

Para conocer mejor el uso de tipos dependientes, veamos la definición del tipo \textit{vector} en  \textit{Idris} \cite{brady:idris-jfp13}, que representa listas cuyo largo es anotado en su tipo:

\begin{code}
data Vect : Nat -> Type -> Type where
  [] : Vect 0 A
  (::) : (x : A) -> (xd : Vect n A) -> Vect (n + 1) A
\end{code}

Un valor del tipo \texttt{Vect 1 String} es una lista que contiene 1 string, mientras que un valor del tipo \texttt{Vect 10 String} es una lista que contiene 10 strings. En la definición \texttt{Vect : Nat -> Type -> Type} el tipo mismo queda parametrizado por un natural, además de un tipo cualquiera.

Tener valores en el tipo permite poder restringir el uso de funciones a determinados tipos que tengan valores específicos. Como ejemplo se tiene la siguiente función:

\begin{code}
head : Vect (n + 1) a -> a
head (x :: xs) = x
\end{code}

Esta función obtiene el primer valor de un vector. Es una función total, ya que restringe su uso solamente a vectores que tengan un largo mayor a 0, por lo que el vector al que se le aplique esta función siempre va a tener por lo menos un elemento para obtener. Si se intenta llamar a esta función con un vector sin elementos, como \texttt{head []}, la llamada no va a compilar porque el typechecker no va a poder unificar el tipo \texttt{Vect 0 a} con \texttt{Vect (n + 1) a} (no puede encontrar un valor natural \texttt{n} que cumpla \texttt{n + 1 = 0}, por lo que falla el typechecking).

En relación a records extensibles, los tipos dependientes hacen posible que dentro del tipo del record puedan existir valores para poder ser manipulados, como podría ser la lista de etiquetas del record. Como esta información se encuentra en el tipo del record, es posible definir tipos y funciones que accedan a ella y la manipulen para poder definir todas las funcionalidades de records extensibles.

Otra propiedad de los tipos dependientes es que permiten definir predicados como tipos. Es posible definir predicados como tipos, y probar tales predicados construyendo valores de ese tipo. Como ejemplo, se puede codificar la propiedad \textit{'Un record extensible no puede tener campos repetidos'} como un tipo \texttt{RecordSinRepetidos record} (donde \texttt{record} es un valor de tipo record), y luego se puede usar un valor \texttt{prueba : RecordSinRepetidos record} en cualquier lugar donde uno quiera que se cumpla esa propiedad. Como se verá más adelante en el trabajo, esto permite tener chequeos y restricciones sobre la construcción de records en tiempo de compilación, ya que si un tipo que representa una propiedad no puede ser construido (lo que significa que no se pudo probar esa propiedad), entonces el código no va a compilar. Esto es una mejora a la alternativa de que se realice un chequeo en tiempo de ejecución, haciendo que falle el programa, ya que uno no necesita correr el programa para saber que éste es correcto y cumple tal propiedad.

\section{Idris}

En este trabajo se decidió utilizar el lenguaje de programación \textit{Idris} \cite{brady:idris-jfp13} para llevar a cabo la investigación del uso de tipos dependientes aplicados a la definición de records extensibles. Idris es un lenguaje de programación funcional con tipos dependientes, con sintaxis similar a Haskell.

Otro lenguaje que cumple similares requisitos es \textit{Agda} \cite{Norell:2009:DTP:1481861.1481862}. Sin embargo, se decidió utilizar Idris para investigar las funcionalidades de este nuevo lenguaje. Las conclusiones obtenidas en este trabajo podrían ser aplicadas a \textit{Agda} también.

La versión de Idris utilizada en este trabajo es la 0.12.0

\section{Guía de trabajo}

A continuación se describe la organización del resto del documento:

En el capítulo \ref{ch:2} se muestran varias implementaciones de records extensibles en otros lenguajes, mostrando sus beneficios e incovenientes. También se muestra una implementación específica de records extensibles en Haskell.

En el capítulo \ref{ch:3} se presenta la implementación de records extensibles en Idris llevada a cabo en este proyecto, expandiendo en el uso de Idris y las funcionalidades de éste que hicieron posible este trabajo. También se muestran los problemas encontrados en este trabajo y sus respectivas soluciones.

En el capítulo \ref{ch:4} se presenta un caso de estudio de construcción y evaluación de expresiones aritméticas, haciendo uso de lo desarrollado en este trabajo, permitiendo ver cómo es el uso de esta implementación de records extensibles y qué se puede llegar a hacer con ella.

En el capítulo \ref{ch:5} se toman las experiencias del desarrollo del trabajo y el caso de estudio y se analizan los problemas encontrados, el aporte que puede llegar a tener este trabajo sobre el problema de records extensibles, entre otras cosas.

En el capítulo \ref{ch:6} se indican aspectos de diseño e implementación que pueden realizarse sobre este trabajo para mejorarlo, como posibles tareas que surgieron y se pueden llevar a cabo.

En el apéndice se encuentran recursos adicionales, como el código fuente de este trabajo.

