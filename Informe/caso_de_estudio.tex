%%%%%%%%%%%%%%%%%%%%%%%%%%%%%%%%%%%%%%%%%%%%
% Caso de estudio
%%%%%%%%%%%%%%%%%%%%%%%%%%%%%%%%%%%%%%%%%%%%

\chapter{Caso de estudio}
\label{ch:4}

\section{Introducción}

Se decidió aplicar la previa definición de records extensibles en un caso de estudio. Éste consiste en definir un pequeño lenguaje aritmético con variables y constantes, y utilizar records extensibles para poder interpretarlo.

Este lenguaje \textit{Exp} se define de la siguiente manera:
\begin{itemize}
\item Si \textit{n} es un natural, entonces \textit{n} pertenece a \textit{Exp} como un literal.
\item Si \textit{x} es un string, entonces \textit{x} pertenece a \textit{Exp} como una variable.
\item Si \textit{e1} y \textit{e2} pertenecen a \textit{Exp}, entonces \textit{e1 + e2} pertenece a \textit{Exp} como una suma.
\item Si \textit{x} es un string, \textit{n} es un natural y \textit{e} pertenece a \textit{Exp}, entonces \textit{let x := n in e} pertenece a \textit{Exp} como una sustitución de una variable.
\end{itemize}

Este lenguaje permite definir expresiones aritméticas donde se pueden definir valorse numéricos, variables, o sumar expresiones. Los siguientes son ejemplos de expresiones de este lenguaje:

\begin{code}
x
x + 3
let x := 3 in x + 3
let y := 10 in (let x := 3 in x + 3)
\end{code}

En este trabajo se va a implementar tal lenguaje en Idris utilizando tipos dependientes, y se van a utilizar records extensibles para definir el contexto necesario para interpretar una expresión de este lenguaje.

\section{Definición de una expresión}

Una expresión de este lenguaje se define en Idris de la siguiente forma:

\begin{code}
data VarDec : String -> Type where
  (:=) : (var : String) -> Nat -> VarDec var

data Exp : List String -> Type where
  Add : Exp fvs1 -> Exp fvs2 -> 
    IsLeftUnion_List fvs1 fvs2 fvsRes -> 
    Exp fvsRes 
  Var : (l : String) -> Exp [l]
  Lit : Nat -> Exp []
  Let : VarDec var -> Exp fvsInner -> 
    DeleteLabelAtPred_List var fvsInner fvsOuter -> 
    Exp fvsOuter
\end{code}

Un valor de tipo \texttt{VarDec x} contiene la variable \texttt{x} y un natural. Representa la declaración de una variable, como \texttt{'x' := 10 : VarDec 'x'}.

El tipo \texttt{Exp} se define como un tipo parametrizado por una lista de variables. Esta lista representa la lista de variables libres que ocurren en la expresión. Un valor de tipo \texttt{Exp ['x', 'y']} es una expresión que tiene a \textit{'x'} e \textit{'y'} como variables libres. Se decidió incluir la información de variables libres en el tipo para poder facilitar tanto la implementación del intérprete, como la construcción de expresiones.

Cada constructor de \texttt{Exp} representa cada ítem de la definición del lenguaje:
\begin{itemize}
\item \texttt{Lit} define una literal numérica. La expresión resultante es \texttt{Exp []} porque no contiene variables libres. 
\item \texttt{Var} define una variable. La expresión resultante contiene esa variable como libre.
\item \texttt{Add} define la suma de dos expresiones. \texttt{IsLeftUnion\_List fvs1 fvs2 fvsRes} es un predicado que indica que \texttt{fvsRes} es la unión por izquierda de la lista \texttt{fvs1} y \texttt{fvs2}. Por lo tanto la expresión resultante tiene como variables libres la unión por izquierda de las variables libres de la primera expresión y de la segunda.
\item \texttt{Let} define la sustitución de una variable. Toma una declaración de variable del estilo \textit{'x' := 10} y una subexpresión. \texttt{DeleteLabelAtPred\_List var fvsInner fvsOuter} es un predicado que indica que la lista \texttt{fvsOuter} es el resultado de tomar la lista \texttt{fvsInner} y eliminarle la variable \texttt{var} (si \texttt{var} no existe en \texttt{fvsInner} entonces la lista resultante es idéntica a ésta). Esto significa que las variables libres de una sustitución son las variables libres de la subexpresión menos la variable que ahora está ligada.
\end{itemize}

\subsection{Construcción de una expresión}

Para poder construir expresiones de este lenguaje, se definieron funciones auxiliares que construyen valores del tipo \texttt{Exp}.

\begin{code}
var : (l : String) -> Exp [l]
var l = Var l

lit : Nat -> Exp []
lit n = Lit n

add : Exp fvs1 -> Exp fvs2 -> Exp (leftUnion fvs1 fvs2)
add {fvs1} {fvs2} e1 e2 = Add e1 e2
  (fromLeftUnionFuncToPred {ls1=fvs1} {ls2=fvs2})

eLet : VarDec var -> Exp fvs -> Exp (deleteAtList var fvs)
eLet {var} {fvs} varDec e = Let varDec e 
  (fromDeleteLabelAtListFuncToPred {l=var} {ls=fvs})
\end{code}

Las cuatro funciones simplemente realizan la aplicación del constructor correspondiente. \texttt{add} y \texttt{eLet} tienen la particularidad de que realizan las computaciones sobre las listas de variables, y construyen los predicados correspondientes dada esas computaciones. \texttt{leftUnion} y \texttt{deleteAtList} son las computaciones análogas a \texttt{IsLeftUnion\_List} y \texttt{DeleteAtLabel\_List}, cumpliendo con los siguientes lemas:

\begin{code}
fromDeleteLabelAtListFuncToPred : DecEq lty => {l : lty} -> 
  {ls : List lty} -> 
  DeleteLabelAtPred_List l ls (deleteAtList l ls)
fromLeftUnionFuncToPred : DecEq lty => {ls1, ls2 : List lty} -> 
  IsLeftUnion_List {lty} ls1 ls2 (leftUnion ls1 ls2)   
\end{code}

A continuación se muestran ejemplos de expresiones construidas con las funciones anteriores:
\begin{code}
exp1 : Exp ["x", "y"]
exp1 = add (var "x") (add (cons 1) (var "y"))

exp2 : Exp ["y"]
exp2 = eLet ("x" := 10) $ add (var "x") (var "y")

exp3 : Exp []
exp3 = eLet ("y" := 5) exp4
\end{code}

También se decidió incluir otra función útil para construir expresiones. Se trata de \textit{local}, que permite tomar una lista de declaraciones de variables y definir la sustitución de todas en una subexpresión:

\begin{code}
data LocalVariables : List String -> Type where
  Nil : LocalVariables []
  (::) : VarDec l -> LocalVariables ls -> 
    LocalVariables (l :: ls) 

localPred : (vars : LocalVariables localVars) -> 
  (innerExp : Exp fvsInner) -> {isSet : IsSet localVars} -> 
  Exp (projectRightList localVars fvsInner)

local : (vars : LocalVariables localVars) -> 
  (innerExp : Exp fvsInner) -> 
  TypeOrUnit 
    (isSet localVars) 
    (Exp (projectRightList localVars fvsInner))
local {localVars} {fvsInner} vars innerExp = 
  mkTypeOrUnit (isSet localVars)
    (\localIsSet => localPred vars innerExp {isSet=localIsSet}) 
\end{code}

\texttt{LocalVariables} es un tipo que contiene una lista de declaraciones locales de variables, y en su tipo mantiene la lista de variables declaradas. Por ejemplo, se tiene \texttt{['x' := 10, 'y' := 4] : LocalVariables ['x', 'y']}.

\texttt{localPred} toma la declaración de variables locales, una subexpresión, una prueba de que las variables declaradas no se repiten, y retorna una nueva expresión. Su implementación se encuentra en el anexo, pero en términos simples construye \texttt{Let}s secuencialmente para cada variable declarada en la lista. La prueba \texttt{IsSet localVars} de que las variables no son repetidas nunca es utilizada, simplemente se agrega para restringir el uso de \texttt{local} y no poder incluir variables repetidas (lo cual sí es posible en aplicaciones sucesivas de \texttt{Let}).

\texttt{local} utiliza el mismo truco definido para \texttt{Record} anteriormente para poder generar la prueba de \texttt{IsSet localVars} de forma automática.

Con estas definiciones se pueden construir expresiones de la siguiente manera:

\begin{code}
exp1 : Exp []
exp1 = local ["x" := 10] $ cons 1

exp2 : Exp []
exp2 = local ["x" := 10, "y" := 9] $ add (var "x") (var "y")
\end{code}


\section{Interpretación de una expresión}

PENDIENTE
